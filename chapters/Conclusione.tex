\chapter*{Conclusione}

In conclusione, con questo lavoro di tesi si è voluta dimostrare l'efficacia dell'algoritmo proposto, che certamente potrà essere migliorato dai futuri lettori, ma che rappresenta un punto di partenza per città più efficienti, meno caotiche e soprattutto meno inquinate, a fronte di un riutilizzo delle tecnologie esistenti e senza ripensare gli incroci cittadini.

Visti i risultati incoraggianti, il passo successivo è quello di realizzare un prototipo funzionante, con delle telecamere da collegare ai semafori di un incrocio, che inviino le loro acquisizioni ad un calcolatore locale, il quale mediante algoritmi di Computer Vision riesca a contare le automobili per ogni corsia, implementando così l'algoritmo e schedulando l'impianto semaforico.

Fatto questo, l'idea è quella di includere tutti i tipi di giunzione, non solo quella a raso a quattro bracci, creando un algoritmo universale che, mediante AI e Machine Learning, riesca autonomamente a riconoscere il tipo di giunzione che sta gestendo e come schedulare al meglio i semafori che ne fanno parte.


Dunque la speranza è che questo progetto rappresenti un punto di partenza per futuri sviluppi, e che possa trovare sbocchi applicativi reali, compiendo, nel suo piccolo, un ulteriore passo verso la trasformazione digitale delle città in cui tutti viviamo.
