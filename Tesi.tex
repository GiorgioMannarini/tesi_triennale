\documentclass[12pt,a4paper]{report}																%	Documento A4, corpo 12pt
\usepackage{multirow}
\usepackage[utf8]{inputenc}																		%	Codifica UTF8
\usepackage[italian]{babel}																		% Lingua italiana
\usepackage{csquotes}
\usepackage{amsmath}																			% Funzioni matematiche
\usepackage{amsfonts}																			% Font personalizzati
\usepackage{amssymb}																			% Simboli
\usepackage{graphicx}																			% Immagini
\usepackage{setspace}																			% Modifica interlinea
\usepackage[left=3cm,right=3cm,top=3cm,bottom=3cm]{geometry}										% Impostazione margini
\usepackage{sectsty}																			% Impostazione titoli capitoli e sezioni
\usepackage{float}																				% Per usare option 'H' sulle figure
\usepackage[font=scriptsize,labelfont=bf]{caption}												% Impostazione didascalie
\usepackage{xcolor}																				% Didascalia più articolata
\usepackage{listings}																			% Inserimento codici sorgente
\usepackage{array}
\newcolumntype{L}[1]{>{\raggedright\let\newline\\\arraybackslash\hspace{0pt}}m{#1}}
\newcolumntype{C}[1]{>{\centering\let\newline\\\arraybackslash\hspace{0pt}}m{#1}}
\newcolumntype{R}[1]{>{\raggedleft\let\newline\\\arraybackslash\hspace{0pt}}m{#1}}
\usepackage[
backend=bibtex,
style=numeric,
sorting=none
]{biblatex}																						% Bibliografia						

\addbibresource{Bibliografia.bib}

% Ridefinizione delle didascalie per i codici sorgente
\renewcommand{\lstlistingname}{Codice}
\renewcommand{\lstlistlistingname}{Elenco dei codici}

% Impostazioni globali per sezioni di codici sorgente
\lstset{ 
	language=Matlab,                					%	choose the language of the code
	%	basicstyle=10pt,       						%	the size of the fonts that are used for the code
	numbers=left,                  					%	where to put the line-numbers
	numberstyle=\footnotesize,      					%	the size of the fonts that are used for the line-numbers
	stepnumber=1,                   					%	the step between two line-numbers. If it's 1 each line will be numbered
	numbersep=5pt,                  					%	how far the line-numbers are from the code
	%	backgroundcolor=\color{white},  				%	choose the background color. You must add \usepackage{color}
	showspaces=false,               					%	show spaces adding particular underscores
	showstringspaces=false,         					%	underline spaces within strings
	showtabs=false,                 					%	show tabs within strings adding particular underscores
	frame=single,	                					%	adds a frame around the code
	% tabsize=2,                						%	sets default tabsize to 2 spaces
	captionpos=b,                   					%	sets the caption-position to bottom
	breaklines=true,                					%	sets automatic line breaking
	breakatwhitespace=false,        					%	sets if automatic breaks should only happen at whitespace
	escapeinside={\%*}{*)},          				%	if you want to add a comment within your code
	extendedchars=true,
	literate=
		{è}{{\`e}}1
		{ù}{{\`u}}1
		{ò}{{\`o}}1
		{à}{{\`a}}1
		{ì}{{\`i}}1
}

% Dimensione titolo dei capitoli
\chapternumberfont{\LARGE} 
\chaptertitlefont{\Large}

% Rimozione indentazione inizio paragrafo
\setlength{\parindent}{0in}

% Interlinea 1.5
\onehalfspace

% Global path per le immagini (EPS)
\graphicspath{{figures/}}

% Inizio documento
\begin{document}
	
\begin{titlepage}
		\begin{center}
			
			{\includegraphics[width=4cm]{logo.pdf}}
			\vspace{0.5cm}
			
			{\LARGE{Politecnico di Bari}}\\
			\vspace{0.5cm}
			
			{\large{Dipartimento di Ingegneria Elettrica e dell'Informazione}}\\
			{\normalsize{Corso di Laurea Triennale in Ingegneria Informatica e dell'Automazione}}\\
			\line(1,0){350}\\
			\vfill
			{\large Tesi in Controllo Digitale}\\
			\vspace{0.5cm}		
			{\large \textbf {
Realizzazione ed implementazione di un algoritmo di controllo di semafori per la gestione ottimizzata del traffico in un modello a più incroci}}\\
		\end{center}
		\vfill
		\begin{minipage}{\textwidth}
			\begin{minipage}{2in}
				\textbf{Relatore:} \\
				Prof. Paolo Lino
			\end{minipage}
			\hfill
			\begin{minipage}{1.3in}
				\textbf{Studente:} \\
				Giorgio Mannarini
			\end{minipage}
		\end{minipage}
		\vfill
		\centering \small {Anno Accademico 2019 - 2020}

\end{titlepage}
% \input{Abstract}
% \input{Dedication}

% Indice, elenco figure, elenco codici
\tableofcontents
\listoffigures
\lstlistoflistings
\listoftables

% Divisione dei capitoli con le relative sezioni
\chapter*{Introduzione}
\addcontentsline{toc}{chapter}{Introduzione}
Il controllo del traffico automobilistico mediante algoritmi performanti, soprattutto per quanto concerne le città più popolose, è un problema affrontato in numerosi documenti di stampo scientifico; tuttavia pochi sono stati i risultati finora raggiunti, ed il sistema più utilizzato per gestire il flusso di veicoli nelle intersezioni stradali resta quello dei semafori staticamente programmati, indipendenti cioè da parametri o eventi esterni.

Certamente negli anni si è compresa quantomeno l’importanza di assegnare una differente priorità, comunque statica, alle varie strade, in modo tale da privilegiare le arterie principali con meccanismi come le onde verdi, tuttavia sarebbe più auspicabile un processo a priorità dinamica, che conceda il verde alle strade più congestionate. Numerose sono le ricerche effettuate in tal senso, sia per quanto concerne l’algoritmo da adottare, che per quanto riguarda la sua relativa implementazione.

In questo lavoro di tesi si è selezionato uno degli algoritmi più comuni, lo si è modificato per migliorarne ulteriormente le performance, e lo si è applicato ad uno scenario di simulazione reale, anch’esso realizzato in maniera autonoma. È da chiarire che l’obiettivo non è stato quello di snellire il traffico quando tutte le arterie che convergono nell’incrocio sono sature di veicoli, quanto più di ottenere buone performance in questa situazione, migliorando molto però il comportamento dei semafori in condizioni di squilibrio, ovvero quando le strade affollate sono solo alcune delle confluenti.

Nello specifico, in primo luogo è stato necessario stabilire la tipologia di incrocio da analizzare, e in tal senso è stata scelta una intersezione a raso a quattro bracci \cite{incrocio}, in cui ogni direzione è percorribile sia in un senso che in quello opposto, e nessuna svolta è vietata. A tal proposito, si è reso fondamentale creare il modello che simula il singolo incrocio in Matlab, Simulink e SimEvents, per monitorare l’andamento del traffico variando alcuni parametri come il tasso di arrivo dei veicoli e la durata del verde, iniziando da un classico algoritmo di gestione statica, implementato comunemente in questo tipo di giunzioni.

Il passo successivo è stato quello di stabilire quali sono le direzioni fra loro compatibili, alle quali cioè è possibile concedere il verde contemporaneamente senza causare incidenti. È da sottolineare che questo lavoro viene normalmente svolto durante il posizionamento e la regolazione di un semaforo classico, e che per ogni tipo di incrocio esistono tabelle indicanti proprio tutte queste direzioni compatibili. La creazione di una tabella di questo genere è il primo passo per la definizione di un algoritmo di gestione ottimizzata dell’incrocio stesso.

Per quanto concerne l’algoritmo ideato, esso si basa su una proposta di \textit{Maram Bani Younes} e \textit{Azzedine Boukerche} \cite{itlc}, modificata ed opportunamente adattata. Si è proceduto dapprima a tradurre l’idea in codice, applicandola ad un singolo incrocio, sfruttando il modello realizzato in precedenza, per poi proseguire con la realizzazione di un modello più complesso e a più ampio respiro, che coinvolge nove diverse intersezioni, collegate fra loro, per simulare una situazione in cui le automobili che lasciano una giunzione confluiscano nella successiva.

A tal proposito, si è reso opportuno modificare ancora una volta l’algoritmo, per adattarlo ad una situazione di questo genere, ed i dati ricavati sono stati confrontati con quelli provenienti dallo stesso modello a nove incroci, gestito però con una turnazione classica dei semafori.

Nel \textbf{capitolo 1} viene dunque introdotto il primo modello realizzato, quello del singolo incrocio, e ne vengono analizzate le performance al variare del tasso di generazione delle auto e alla durata del verde per ciascuna strada, senza l’utilizzo di un algoritmo a priorità dinamica.

Nel \textbf{capitolo 2} viene esplicato il funzionamento dell’algoritmo sopracitato, le modifiche apportate, e viene mostrato come la sua applicazione al modello precedente, a parità di altre condizioni, migliori le performance dell’incrocio, misurate in termini di numero di auto in coda e tempo di attesa medio delle stesse.

Nel \textbf{capitolo 3} viene presentato il secondo modello realizzato, relativo ad una situazione con nove incroci interconnessi, e ne vengono analizzate le prestazioni in assenza di un algoritmo di controllo.

Nel \textbf{capitolo 4}, in fine, si analizza il comportamento di tale modello in presenza di un algoritmo che tenga conto del numero di auto in coda in ciascuna corsia, e di altri parametri esterni, confrontando i risultati ottenuti con quelli del capitolo precedente.


% Inizio capitolo
\chapter{Modellazione e gestione statica del singolo incrocio \label{capitolo1}}
Come accennato nell’introduzione, è stato scelto di analizzare un incrocio a raso a quattro bracci, senza limitazioni in termini di svolte e direzioni percorribili. Questo significa che la giunzione, composta appunto da quattro strade disposte nella classica forma a croce, presenta quattro corsie per ogni strada. Per comodità e per adottare una convenzione di linguaggio, ci si riferirà alle automobili che non hanno ancora superato l’incrocio definendole “entranti”, chiamando invece “uscenti” le vetture che già sono passate attraverso la giunzione.

Le quattro corsie, dunque, sono così assegnate: due sono dedicate a veicoli uscenti, provenienti cioè da una delle altre tre strade dell’incrocio, mentre le altre due sono dedicate alle automobili entranti. Nello specifico, adottando una convenzione di “guida a destra”, comune nella maggior parte dei paesi europei, la corsia più a destra delle due entranti è riservata a chi vuole svoltare a destra o andare dritto, mentre quella più a sinistra a coloro i quali vogliono svoltare a sinistra.

Questa suddivisione è molto comune negli incroci italiani, ed è la soluzione più utilizzata nella assegnazione dei sensi di marcia in incroci di questo genere.

Le corsie a cui può essere concesso il verde contemporaneamente sono sempre e solo due, e per ogni corsia ne esistono due possibili complementari, che possono cioè essere scelte per ottenere il verde assieme alla corsia suddetta. La figura in basso è riassuntiva di questo modello, che è applicabile alla tipologia di incrocio scelto ed è indipendente dal tipo di algoritmo utilizzato per controllare i semafori.

Nella gestione ordinaria degli incroci stradali solo una delle due alternative è scelta, in maniera statica, per ogni corsia, in base al flusso di traffico che si prevede per ciascun senso di marcia.

\begin{figure}[H]
	\centering
	\includegraphics[width=1\textwidth]{compatibili.pdf}
	\caption{Direzioni compatibili}
	\label{dirCompatibili}
\end{figure}
\label{figure: DirezioniCompatibili}
Nel modello oggetto di questo primo capitolo, è stato scelto di concedere il verde alle due corsie facenti parte della stessa strada, quella per svoltare a sinistra e quella per svoltare a destra / andare dritto, ovviamente a turno, per ognuno dei quattro bracci dell’incrocio (schemi (b), (c), (f), (g)). Ciò tuttavia non esclude che, essendo il sopracitato modello profondamente personalizzabile e parametrizzato, si possa prevedere anche la seconda opzione, indicata dagli schemi (a), (d), (e), (h) in figura.

\newpage
\section{Generazione delle automobili}
Per quanto concerne la generazione delle auto, questa è indipendente per ogni corsia ed avviene mediante un processo di Poisson, utilizzato nella maggior parte dei modelli di simulazione del traffico \cite{probability}. Se di tale processo stocastico si è interessati alla probabilità che il tempo trascorso tra due eventi consecutivi sia pari a t è necessario fare ricorso ad una distribuzione esponenziale, di parametro t, la cui funzione densità di probabilità $ f(t)$ in questione, nello specifico, è la seguente.

\begin{equation}
PDF: f(t) = \lambda e^{\lambda t}
\end{equation}
Gli Entity Generator di Simulink, gli oggetti utilizzati per generare le automobili, prevedono tuttavia che la funzione che li governa sia riferita a $dt$, ovvero all’intervallo di tempo fra due eventi consecutivi, che deve essere una variabile casuale, e non alla sua funzione densità di probabilità. Pertanto, per ricavare tale intervallo di tempo dalla funzione densità di probabilità in oggetto, si fa riferimento alla sua funzione di distribuzione $F(t)$, ottenuta integrando $f(t)$ nell'intervallo $[0, t]$ e che rappresenta la probabilità che in tale intervallo si sia verificato un evento, in questo caso l’arrivo di un’automobile. Tale funzione di distribuzione ha la seguente espressione.
\begin{equation}
CDF: F(t)  = 1-e^{-\lambda t}
\end{equation}
Da essa è facile ricavare la variabile $t$, che rappresenta l’incognita del problema, come inversa della $CDF$.
\begin{equation}
CDF^{-1}: t = -\frac{1}{\lambda} \ln{[1 - F(t)]}
\end{equation}
Nello specifico $F(t)$, che, come già detto, è la probabilità che in un certo intervallo di tempo si verifichi un evento, è scelta casualmente mediante una distribuzione uniforme, fra 0 e 1. Pertanto, se $F(t)=1$ si ha la certezza che c’è stato un arrivo, e questo può essere vero solo se $t \to +\infty$, al contrario per $F(t)=0$ si è sicuri che non si sia verificato alcun evento, e ciò avviene solo se $t \to 0$.

Si nota, dunque, che più aumenta $t$, più è probabile che arrivi un’auto nella corsia in questione. Il tutto è governato dal parametro $\mu = \frac{1}{\lambda}$, arbitrariamente impostabile.

Tale funzione è stata quindi implementata negli Entity Generator, che come già accennato si occupano di generare, appunto, entità, oggetti con determinate caratteristiche controllate dai due toolbox, Simulink e SimEvents.

Per rendere ancora più variabile e casuale il tempo medio che intercorre fra l’arrivo di un’auto e quello della successiva, il parametro $\mu$, e dunque il tasso di generazione delle vetture viene originato randomicamente ad ogni esecuzione, in un intervallo precedentemente determinato, utilizzando questa volta una distribuzione uniforme, capace dunque di assicurare assoluta equi-probabilità. Inoltre, poiché si prevede che durante le ore del giorno il numero di veicoli vari sensibilmente fra le ore di punta, le ore notturne e quelle intermedie, è stato anche previsto un meccanismo per permettere a SimEvents di rilevare in quale ora del giorno si trova durante la simulazione, e aumentare o diminuire di conseguenza $\mu$. In definitiva, il codice utilizzato per la generazione delle automobili è alla seguente pagina.
\newpage
\begin{lstlisting}[language=Matlab,label=entity-generator,caption=Entity Generator]
%Random number generation

coder.extrinsic('rand');
ValEntry1 = 1;
ValEntry1 = rand();
%ValEntry1 rappresenta la CDF

index = 1;
%indice del generator, in questo caso il primo.

% Ora del giorno
time = mod(getTime(), 86400) / 3600;

% Fasce orarie
if time > 0 && time < 7
myMu = muNotte(index);
elseif time > 7.5 && time < 9
myMu = muMattinaPranzo(index);
elseif time > 12 && time < 14.5
myMu = muMattinaPranzo(index);
elseif time > 18 && time < 20
myMu = muSera(index);
else
myMu = muArray(index);
end

% Pattern: distribuzione esponenziale 
dt = -myMu * log(1 - ValEntry1);
% dt: tempo fra due eventi di Poisson
\end{lstlisting}
\newpage
Come è facile comprendere, si ha, rispetto alla descrizione precedente, $\mu = myMu$.

È inoltre da precisare che i parametri $muNotte$, $muMattinaPranzo$, $muSera$ e $muArray$ sono definiti come variabili globali, con questo codice.

\begin{lstlisting}[language=Matlab,label=mu-definition,caption= Tassi di generazione variabili]
coder.extrinsic('rand');
temp = 1;
temp = rand([1 8]);
%creo 8 numeri casuali fra 0 e 1, uno per ogni generator
muArray = 20 + (35 - 20) * temp;
%l'intervallo scelto va da 20 a 35

muNotte = muArray + 80; %di notte il tasso è molto basso
muMattinaPranzo = muArray - 6; %di mattina ed a pranzo è molto alto
muSera = muArray - 3; %di sera è più alto del normale
\end{lstlisting}

Per definire delle variabili globali è necessario accedere al \textit{Model Workspace} mediante il pulsante in basso a sinistra in \textit{figura \ref{fig:accesstoworkspace}}. Una volta fatto ciò, spostandosi sul modello (in questo caso \textit{GateControl}), e sulla tab \textit{Callbacks} in alto a destra, è possibile inserire il codice appena presentato cliccando su \textit{InitFcn}, ovvero lo spazio in cui possono essere dichiarate le funzioni / variabili di inizializzazione del modello.

\begin{figure}[H]
  \includegraphics[width=1\textwidth]{accessToWorkspace.pdf}
  \caption{Accesso al workspace, ingrandimeto}
  \label{fig:accesstoworkspace}
\end{figure}

\begin{figure}[H]
  \includegraphics[width=1\textwidth]{InitFcn.pdf}
  \caption{Inserimento di variabili globali e codice di inizializzazione nel modello}
  \label{fig:initfcn}
\end{figure}

Si può notare dal \textit{codice \ref{mu-definition}} che vengono generati otto numeri casuali, fra 20 e 35, che costituiscono i valori base per il parametro $\mu$. Ciò vuol dire che mediamente fra l’arrivo di un’auto e della successiva intercorre un tempo compreso tra:
\begin{equation*}
	dt = -20 \ln{[1 - 0.5]} = 13.86s
\end{equation*} 
\begin{center}
	e
\end{center}
\begin{equation*}
dt = -35 \ln{[1 - 0.5]} = 24.26s
\end{equation*}
dove 0.5 è il valore medio della distribuzione uniforme precedentemente descritta. Tali valori base sono utilizzati nelle ore intermedie, come già detto, mentre sono modificati durante le ore di punta e quelle notturne.

Il valore di $\mu$ è poi profondamente decrementato durante le ore di punta (mattina e ora di pranzo), incrementato durante la notte e decrementato, anche se meno drasticamente, all’ora di cena, influendo ovviamente sui tempi di interarrivo, che diventano i seguenti.

\textbf{Mattina e pranzo} (fra le ore 7.30 e le ore 9.00, e fra le ore 12.00 e le ore 14.30): da 9.7 secondi a 20.1 secondi.

\textbf{Sera} (fra le ore 18.00 e le ore 20.00): da 11.78 secondi a 22.18 secondi.

\textbf{Notte} (fra le ore 0.00 e le ore 7.00): da 69.31 secondi a 79.71 secondi.

Il tutto è stato fatto per rendere il modello quanto più imprevedibile possibile, mantenendolo però fedele alla realtà.

Si nota che, con questi tassi, vengono mediamente generate 2200 automobili per ogni corsia, ovvero circa 17600 vetture attraversanti l’incrocio durante l’arco di 24 ore, come riportato dal seguente grafico.

\begin{figure}[H]
	\centering
	\includegraphics[width=1\textwidth]{AutoTotaliStatico.pdf}
	\caption{Vetture complessive nell'arco di 24 ore per una singola corsia}
\end{figure}
\newpage

La scelta di questi valori non è stata casuale: si è voluto infatti considerare il caso di un’intersezione affollata, plausibile nelle grandi città. Variando il parametro manualmente e sperimentando rispetto al numero di automobili, si è trovato in questi numeri un buon compromesso per analizzare il comportamento del modello e dell’algoritmo applicato. 

È infatti da notare che negli incroci in cui mediamente il numero di veicoli giornalieri è molto basso l’applicazione di un algoritmo di controllo come quello che sarà presentato nel capitolo successivo di questo lavoro di tesi risulta sì essere utile, ma meno drasticamente determinante, in quanto non vi è una situazione di traffico intenso e di profondo squilibrio, con strade molto affollate e strade con poche automobili, situazione che invece verrà ricreata, in seguito, per confrontare l’algoritmo statico di questo capitolo con quello dinamico, di cui si discuterà. 

Anche le variazioni in base alle fasce orarie sono state analizzate, e non lasciate al caso: si è voluto infatti simulare un pesante congestionamento nelle ore del giorno mediamente più critiche per una giunzione strategica di una città, ossia fra le ore 7.30 e le ore 9.00, quando gli alunni prendono posto in classe ed i lavoratori arrivano nei loro uffici, così come fra le ore 12.00 e le ore 14.30, e in serata, fra le ore 18.00 e le ore 20.00. Si è considerato, tuttavia, che in quest’ultima fascia oraria, complice l’assenza dei bambini, ormai a casa, e di molti anziani, il tasso sia sì più elevato rispetto a quello standard, per via dei lavoratori, di ritorno alle loro abitazioni, ma meno rispetto a quello mattutino e dell’ora di pranzo.
Durante la notte, invece, si è voluto mantenere il numero di veicoli molto basso, anche questa volta facendo riferimento ad un grande incrocio cittadino. 

Quindi in definitiva i valori di $\mu$ sono frutto di simulazioni sperimentali e di un compromesso fra una situazione realmente plausibile ed un congestionamento dell'incrocio comunque elevato.

Per concludere, la seguente tabella è riassuntiva rispetto ai valori di $\mu$ scelti in base all’ora del giorno ed ai relativi tempi di inter-arrivo. Nel paragrafo conclusivo di questo capitolo, a tale tabella verranno aggiunti anche i tempi medi di attesa di ciascuna automobile nonché il numero medio di auto in coda per ciascuna corsia, e sarà certamente più chiara la corrispondenza fra il parametro ed il flusso di autovetture.
\newpage

\begin{table}[H]
\centering
\begin{tabular}{|c|c|C{3.7cm}|C{3,7cm}|}
  \hline
  \textbf{Fascia Oraria} & 
  \textbf{Intervallo $\mu$} & 
  \textbf{Stima tempo minimo fra due arrivi} & 
  \textbf{Stima tempo massimo fra due arrivi} \\\hline
  0.00 - 7.00 & 
  100 - 115 & 
  69.31s & 
  79.71s \\\hline
  7.00 - 7.30 & 
  20 - 35 & 
  13.86s & 
  24.26s \\\hline
  7.30 - 9.00 & 
  14 - 29 & 
  9.7s & 
  20.1s \\\hline
  9.00 - 12.00 & 
  20 - 35 & 
  13.86s & 
  24.26s \\\hline
  12.00 - 14.30 & 
  14 - 29 & 
  9.7ss & 
  20.1s \\\hline
  14.30 - 18.00 & 
  20 - 35 & 
  13.86s & 
  24.26s \\\hline
  18.00 - 20.00 & 
  17 - 32 & 
  11.78s & 
  22.18s \\\hline
  20.00 - 0.00 & 
  20 - 35 & 
  13.86s & 
  24.26s \\\hline
\end{tabular}
\caption{Relazione fra il parametro $\mu$ ed i tempi di interarrivo}
\label{table:tabellatempiinterarrivo}
\end{table}
È importante anche indicare la presenza di lassi temporali intermedi, fra un cambiamento ed un altro, come quello relativo alla fascia oraria 7.00 – 7.30, in cui il valore di $\mu$ viene posto al valore standard, che rappresenta a tutti gli effetti una “via di mezzo” fra la stima precedente e la successiva. Questo per non provocare un cambiamento troppo repentino del tasso di arrivo dei veicoli, inverosimile nella realtà, in cui aumenta e diminuisce gradualmente. 


\newpage
\section{Il modello in Simulink e SimEvents: introduzione}
Per quanto concerne il resto del modello, come si può notare nella figura seguente, ogni strada è rappresentata, oltre che dall’Entity Generator relativo, da una coda di tipo FIFO (First In First Out), la cui capienza può essere variata a piacimento, da un Entity Gate, che rappresenta il semaforo, e che dunque può assumere lo stato di \textit{chiuso} (semaforo rosso) o \textit{aperto} (semaforo verde), da un Entity Server, utile per modellare il tempo che una singola macchina trascorre nell’incrocio vero e proprio, quindi da quando lascia la strada da cui proviene a quando imbocca quella di uscita, ed in fine da un Entity Terminator, che non fa altro che eliminare dal modello i veicoli che hanno già lasciato l’incrocio.

\begin{figure}[H]
	\centering
	\includegraphics[width=0.75\textwidth]{ModelloSingoloIncrocio.pdf}
	\caption{Modello di un incrocio a raso a quattro bracci in Simulink e SimEvents, gestione statica}
	\label{fig: modelloSingoloIncrocio}
\end{figure}
\newpage
Sono da fare alcune importanti precisazioni:
\begin{enumerate}
	\item La grandezza delle code è stata impostata ad infinito, questo per valutare l’efficienza dei due algoritmi e non avere risultati falsati dati dal fatto che le code si riempiono. In altre parole, non ci sono limiti, in questo modello, a quante macchine possono essere presenti in una singola corsia, tuttavia questi limiti possono essere tranquillamente imposti per poter rendere lo schema più fedele alla realtà. Ciò vale esclusivamente per il modello a singolo incrocio, per i motivi suddetti. Nel Capitolo 3 verrà introdotto un nuovo modello, con più incroci interconnessi, e questo, per adattarsi meglio al funzionamento reale, sarà proposto con code che presentano una capienza massima, condizionata dalla distanza della giunzione in questione da quella precedente.
	\item Il tempo che una macchina trascorre al centro dell’incrocio, definito all’interno degli Entity Server, ha una componente casuale, in quanto si è osservato che anche questo parametro, nel mondo reale, non è fisso, e si è voluta inserire un’ulteriore variabile di imprevedibilità nel modello. Nello specifico si è utilizzata una distribuzione uniforme in un intervallo [4, 6] secondi, come si può notare anche nel \textit{codice \ref{parametri-globali-server}} inerente alla spiegazione specifica del componente in questione. Questi valori non sono casuali: si è voluto stimare il tempo che una macchina impiega ad abbandonare l’incrocio. Considerando che la prima automobile della coda è la più lenta, mentre l’ultima sarà certamente la più veloce, è stato necessario stimare un valore medio. Tale stima resta confinata in un intervallo ristretto perché si è voluto mantenere un valore medio di 5s, senza peggiorare il valore limite superiore, essendo piuttosto inverosimile che un’automobile impieghi più di 6s da quando entra nella giunzione a quando ne esce, in assenza di cause esterne.
	\item È da approfondire anche il concetto della luce gialla: il giallo, infatti, indica alle macchine che non hanno ancora superato il semaforo di fermarsi, e a quelle che stanno occupando l’incrocio di sgomberarlo al più presto. Questo, in effetti, è implicitamente implementato: quando l’Entity Gate si chiude, le macchine al centro dell’incrocio lo liberano, essendo già all’interno dell’Entity Server, consentendo poi a quelle provenienti dalle altre corsie di fluire.
\end{enumerate}
\begin{figure}[H]
  \includegraphics[width=1\textwidth]{catena.pdf}
  \caption{Ingrandimento su una singola corsia}
  \label{fig:catenaSingolaCorsia}
\end{figure}
Oltre ai numerosi oscilloscopi presenti, utili a visualizzare l'evoluzione nel tempo di una serie di informazioni, come il numero di macchine in ogni coda, la durata del verde per ciascun semaforo, il numero di macchine totali che hanno attraversato ogni corsia, il tempo medio d’attesa di ogni corsia, è da sottolineare che l’intero incrocio è governato da un blocco di tipo Enabled Subsystem, all’interno del quale è stata inserita una funzione Matlab, in un blocco Function-Call Subsystem. Ciò che questo blocco consente, è di eseguire tale funzione, una sola volta, solo quando viene inviato un segnale di trigger.

In altre parole, la funzione si occupa di cambiare lo stato del sistema (dare il rosso ai semafori attualmente verdi e concedere il verde ai successivi, per ora staticamente), e deve essere eseguita solo quando effettivamente il tempo concesso alla luce verde dalla chiamata precedente si esaurisce.

A tale scopo, il trigger è gestito da un timer, oggetto non presente in alcuna libreria, che si è prodotto in autonomia. Questo timer accetta in ingresso il tempo, in secondi, per il quale deve restituire in uscita 0, che equivale a \textit{trigger inattivo}, capisce da SimEvents il numero di secondi passati dall’ultima chiamata, e ad ogni passo di campionamento (ogni secondo), esegue il confronto fra questo dato e quello in input, riuscendo così a richiamare la funzione ad intervalli temporali desiderati e ben precisi.
Attualmente questo meccanismo sembra esageratamente complesso, in quanto i semafori hanno delle tempistiche prefissate, che saranno più chiare nel \textit{paragrafo \ref{section: AlgoritmoStatico}}, dedicato alla spiegazione dell’algoritmo di questo primo capitolo, tuttavia in previsione dell’implementazione di una politica di decisione dinamica, sia per quanto concerne le direzioni a cui assegnare il verde, che per la durata dello stesso, si è reso necessario adottare questo tipo di oggetto.
\newpage
\section{Il modello in Simulink e SimEvents: analisi dettagliata dei componenti}

Come accennato in precedenza, la \textit{figura \ref{fig:catenaSingolaCorsia}} è riassuntiva di come è stata modellata una singola corsia. Nello specifico verrà proposta in questo paragrafo una spiegazione più dettagliata di ogni componente.
\newline

\textbf{Entity Generator\cite{entitygen}:} serve a generare entità. Le entità in Simulink e SimEvents sono oggetti con determinate caratteristiche, che possono muoversi all’interno del modello fra componenti collegati fra di loro. In questo caso le entità rappresentano le vetture. 

Ogni entità è un oggetto a sé stante ed ha un proprio ciclo di vita: viene generata da un Entity Generator e terminata da un Entity Terminator, descritto in seguito.

Per configurare un Entity Generator bisogna modificare, dopo aver aperto la finestra ad esso relativa con un doppio click, lo script relativo ad “Entity Generation”, in cui è stato inserito il \textit{codice \ref{mu-definition}}. Il metodo di generazione è “Time-Based”, ovvero basato su una distribuzione che restituisce a Simulink un numero, rappresentante del tempo di inter-arrivo. In sostanza ad ogni generazione questa funzione viene richiamata, e restituisce l’intervallo di tempo che intercorrerà fino alla generazione successiva. Fondamentale è la spunta su “Generate entity at simulation start”, che fa partire il tutto. Tutti gli altri parametri sono stati lasciati invariati.
\newline
\begin{figure}[H]
  \includegraphics[width=1\textwidth]{generatorEproprieta.pdf}
  \caption{Entity Generator e relativa configurazione}
  \label{fig:entitygenandprops}
\end{figure}
\newpage

\textbf{Entity Queue\cite{entityqueue}:} il suo scopo è quello di implementare una coda di tipo First In First Out, all’interno della quale convergono le entità. Nel modello, rappresenta la corsia nella quale giungono le vetture, e può essere collegata direttamente all’Entity Generator, in quanto anche questo componente lavora con le entità. Una volta che un’automobile viene generata, essa automaticamente confluisce in questa coda. Anche questa volta, a seguito di un doppio click sono stati configurati alcuni parametri. Nello specifico “Capacity” è stato posto uguale ad una variabile globale, \textit{queueSize}, che in questo modello è pari ad infinito per i motivi già spiegati in precedenza. È importante impostare il “Queue type” a FIFO, e comunicare alla coda che le entità giungeranno mediante la sua porta di ingresso (Entity arrival source pari a Input port). Per collegare oscilloscopi alla coda per visionare informazioni utili ci si può spostare nella tab “Statistics” e spuntare le caselle desiderate.
\newline
\begin{figure}[H]
  \includegraphics[width=1\textwidth]{queueEproprieta.pdf}
  \caption{Entity Queue e relativa configurazione}
  \label{fig:entityqueueprops}
\end{figure}
\newpage
\textbf{Entity Gate\cite{entitygate}:} rappresenta il semaforo. Quando è aperto, le entità sono libere di fluirvi attraverso, esattamente come un semaforo verde. È importante selezionare, nella finestra che si apre cliccando due volte sull’elemento, come modalità operativa “Enable gate”, in modo tale da farlo funzionare nel modo descritto, ovvero come chiuso quando in ingresso riceve uno 0 ed aperto quando riceve un 1. Si è scelto, inoltre, di partire da una situazione in cui tutti i semafori sono rossi, quando il modello viene inizializzato, quindi la casella “Open gate at simulation start” è stata deselezionata. 

L’apertura e la chiusura del gate può essere controllata collegando un elemento di controllo alla porta in alto, che rappresenta il suo trigger. Tale elemento è un output del blocco Enabled Subsystem, di cui si parlerà poco più avanti.
\newline
\begin{figure}[H]
  \includegraphics[width=1\textwidth]{gateEproprieta.pdf}
  \caption{Entity Gate e relativa configurazione}
  \label{fig:entitygateprops}
\end{figure}
\newpage
\textbf{Entity Server\cite{entityserver}:} la sua funzione è quella di rappresentare l’incrocio vero e proprio, da quando una vettura supera il semaforo a quando entra nella corsia di uscita. Come spiegato, un’automobile impiega del tempo per percorrere tale tratto di strada, tempo che appunto viene modellato in questo componente. Con un doppio click sull’Entity Server, come al solito, è possibile modificarne i parametri. Nello specifico si è impostata la capacità di questo elemento ad 1, inserendo il seguente codice che modella la variabilità del tempo che un’automobile passa nell’incrocio. Le variabili \textit{minUniform} e \textit{maxUniform} sono globali, e sono proprio 4 e 6, come spiegato in precedenza e come si può notare nel \textit{codice \ref{parametri-globali-server}}.
Tutte le altre impostazioni non sono state modificate.
\begin{figure}[H]
  \includegraphics[width=1\textwidth]{serverEproprieta.pdf}
  \caption{Entity Server e relativa configurazione}
  \label{fig:entityserverprops}
\end{figure}
\begin{lstlisting}[language=Matlab,label=parametri-globali-server,caption=Codice relativo ai parametri globali dell'Entity Server]
coder.extrinsic('rand');
minUniform = 4;
maxUniform = 6;
\end{lstlisting}
\begin{lstlisting}[language=Matlab,label=codiceentityserver,caption=Codice inserito negli Entity Server]
    coder.extrinsic('rand');
ValEntry1 = 1;
ValEntry1 = rand();
% Pattern: Uniform distribution
dt = minUniform + (maxUniform - minUniform) * ValEntry1;
\end{lstlisting}

\textbf{Entity Terminator\cite{entityterminator}:} serve ad eliminare le entità dal modello, viene collegato all’Entity Server e non è stato modificato in alcun modo.
\newline
\begin{figure}[H]
\center
  \includegraphics[width=0.2\textwidth]{terminatorBlock.pdf}
  \caption{Entity Terminator}
  \label{fig:terminatorblock}
\end{figure}


\textbf{Enabled Subsystem\cite{enabledsub}:} è un blocco “container”, e serve a fare in modo che ciò che è presente al suo interno venga eseguito, una sola volta, quando alla porta di trigger di questo componente arriva un segnale pari ad 1. Viene collegato al timer realizzato, che sarà spiegato più avanti. Cliccandoci due volte è possibile notare che al suo interno è posto un blocco \textbf{Function-Call Subsystem\cite{functioncallsub}}, che contiene poi la funzione di controllo (Matlab) vera e propria.

Tutto questo è stato fatto per garantire che la logica di controllo sia effettivamente richiamata solo allo scadere del timer, ovvero quando l’incrocio deve passare da uno stato ad un altro. Il blocco Function Call Subsystem presenta in uscita gli output della funzione al suo interno, descritti nel paragrafo successivo. Questi output, poi, sono inviati al blocco Enabled Subsystem, che si fa carico di inviarli al modello principale. È importante notare che, poiché nel modello si ragiona con le entità, mentre le uscite di una funzione di Matlab sono delle variabili, bisogna incapsulare tali variabili in entità appunto. Ciò è possibile utilizzando dei blocchi di tipo \textbf{Message Send\cite{messagesend}}, che fanno da intermediari fra le uscite dell’Enabled Subsystem ed i componenti che tali uscite governeranno, ovvero gli Entity Gate, come sarà più chiaro, ancora una volta, nel prossimo paragrafo.

Come si può notare, uno degli output non è inviato al modello principale, ma è incapsulato in un blocco di tipo \textbf{Function-Call Feedback Latch\cite{feedbacklatch}}. Questo output rappresenta lo stato dell’incrocio, che servirà da input alla chiamata successiva per concedere il verde ai semafori secondo una turnazione. È anche interessante notare che l’uscita \textit{delay}, che non è altro che il tempo assegnato alla luce verde, non ha bisogno di essere incapsulata in un messaggio, essendo direttamente utilizzata dal timer.
\newline
\begin{figure}[H]
  \includegraphics[width=1\textwidth]{functionCaller.pdf}
  \caption{Blocchi inerenti alla funzione di controllo}
  \label{fig:blocchicontrollo}
\end{figure}
\begin{figure}[H]
\center
  \includegraphics[width=0.2\textwidth]{message.pdf}
  \caption{Il blocco Message Send}
  \label{fig:messagesend}
\end{figure}
\newpage

\textbf{Il timer realizzato:} anche questo componente è in realtà un sottosistema, ovvero, in un blocco \textbf{Subsystem\cite{subsystem}} è stato realizzato un meccanismo per inviare in uscita il numero 1 allo scadere di un timer. Prima di entrare nel merito di quanto creato, è bene notare che l’uscita non controlla direttamente il trigger del blocco Enabled Subsystem, in precedenza descritto, ma che prima venga ritardata di un secondo grazie a un blocco \textbf{Unit Delay\cite{delay}}. Ciò si è reso necessario per evitare alcuni errori, stabilizzando tale uscita.

È dunque chiaro che il timer in realtà non considera esattamente il tempo inviato in uscita dalla funzione, ma aggiunga ad esso un secondo preventivamente, secondo poi recuperato dall’attesa data da questo blocco.

Cliccando due volte sul sottosistema, si può notare come esso sia alquanto elementare. Sfruttando un contatore e fissando il passo di campionamento ad un secondo, ciò che succede è che ogni secondo tale contatore viene incrementato di 1, e viene confrontato con il valore in ingresso, a cui come precedentemente detto viene sommato 1 secondo. Quando questi due valori sono uguali, ovvero quando la loro differenza è nulla, allora il contatore viene resettato, e viene fornita un’uscita pari ad 1.
\begin{figure}[H]
  \includegraphics[width=1\textwidth]{temporizzatoreESubsystem.pdf}
  \caption{Il blocco timer realizzato}
  \label{fig:timersub}
\end{figure}
\begin{figure}[H]
\center
  \includegraphics[width=0.1\textwidth]{1_z.pdf}
  \caption{Il blocco Unit Delay}
  \label{fig:unitdelay}
\end{figure}
\newpage
\textbf{La funzione getTime:} per poter acquisire l'orario della simulazione si è creato, anche questa volta, un oggetto apposito, incapsulato in un blocco \textbf{Simulink Function\cite{simfunc}}. Nello specifico, si è usato un \textbf{Digital Clock\cite{digclock}}, che restituisce un valore numerico che rappresenta il numero di secondi trascorsi dall'avvio della simulazione. Poiché è stato scelto di far durare tale simulazione 86400s, ovvero un giorno, questo numero oscilla fra lo 0 (corrispondente alla mezzanotte) ed 86400, appunto. Collegando tale blocco all'uscita della Simulink Function, si può richiamare la funzione ogni volta che lo si necessita, comprendendo così come far variare i tassi di generazione in base all'ora del giorno.
\begin{figure}[H]
  \includegraphics[width=1\textwidth]{digitalClock.pdf}
  \caption{Il blocco Simulink Function per la funzione getTime()}
  \label{fig:gettime}
\end{figure}














\newpage
\section{L'algoritmo di gestione statica dell'incrocio} \label{section: AlgoritmoStatico}
In questa prima fase, come precedentemente accennato, si è voluto modellare il comportamento di un incrocio di questo genere in maniera quanto più possibile fedele alla realtà, per avere un metro di paragone per le successive elaborazioni. Nello specifico, con riferimento alla \textit{figura 1.1}, e come già detto, sono state scelte le direzioni (b), (c), (f), (g), da alternare, concedendo a ciascuna una durata del verde pari a 25 secondi, parametro plausibile se si considera un incrocio grande ed affollato, cosa effettivamente vera, viste le circa 17600 macchine giornaliere.

L’algoritmo implementato è il seguente.

\begin{lstlisting}[language=Matlab,label=static-algorithm,caption= Algoritmo di gestione statica del singolo incrocio]
function [delay, y1, y2, y3, y4, y5, y6, y7, y8, current] = fcn(previous)

GreenTime = 25;
% Current e previous indicano a quale strada si sta dando / si è dato il verde 
current = zeros([1 4]);
next = 1; %Elemento del vettore a cui dare il verde
for i = 1 : 1 : 4
if previous(i) == 1
next = mod(i, 4) + 1;
end
end
current(next) = 1;
%Risultato
U = ([0 0; 0 0; 0 0; 0 0]);
U(next, 1) = 1;
U(next, 2) = 1;
delay = GreenTime; y1 = U(1, 1); y2 = U(1, 2); y3 = U(2, 1); y4 = U(2, 2); y5 = U(3, 1); y6 = U(3, 2); y7 = U(4, 1); y8 = U(4, 2);
%Assegnazione delle uscite

end
\end{lstlisting}
\newpage
La funzione accetta in input \textit{previous}, che è un array di quattro elementi (uno per ogni braccio), nel quale l’unico elemento posto ad uno è quello inerente al braccio che ha ricevuto il verde quando lo script è stato chiamato l’ultima volta. Tutti gli altri valori sono pari a zero. È dunque chiaro che in realtà il vettore in questione si comporta come un canale di feedback, in cui ogni elemento è un flag, e serve per gestire la turnazione.

Come è facile notare, infatti, all’interno dello script viene inizializzato un nuovo array, \textit{current}, sempre di quattro elementi. Si valuta poi quale è stato l’ultimo braccio a cui si è concesso il verde, mediante un \textit{ciclo for}, e si fa scattare il semaforo successivo.

Alla successiva chiamata l’attuale array \textit{current} diventerà a tutti gli effetti il relativo \textit{previous}.

È anche da notare, che oltre alle otto uscite (una per ogni corsia), che in realtà in questo caso statico sono sempre accoppiate (ecco il perché di un vettore \textit{previous} di soli quattro elementi), la funzione manda in output anche il \textit{delay}, ovvero l’intervallo di tempo nel quale le macchine delle strade a cui è stato concesso il verde possono fluire liberamente. Tale \textit{delay} viene poi inviato direttamente al timer, che provvederà a richiamare la funzione alla sua scadenza.

Ogni uscita è poi collegata all’Entity Gate della rispettiva corsia, che, ricevendo 0 si chiude (o resta chiuso), ricevendo 1 permette invece ai veicoli di occupare l’incrocio.
\newpage
\section{Analisi dei risultati}
Si analizzano dunque qui i risultati di questa gestione, utili poi per un confronto con l’algoritmo di ottimizzazione oggetto di questa tesi. In particolare, i grafici che seguono sono ottenuti con una simulazione in cui i parametri $\mu$ per ogni corsia sono i seguenti. Tali grafici, inoltre, si riferiscono alla prima corsia (corsia di sinistra del braccio ovest), dunque al primo parametro di questo vettore.

\begin{center}
	[33.7997   32.4750   23.2955   33.1698   29.9311   24.1165   31.6948   28.4784]
\end{center}

Come si può notare dalla \textit{figura \ref{fig:AutoInCodaStatico}}, riferita al numero di auto in coda in funzione dell’ora del giorno considerata, si ha una sensibile variazione fra il numero di auto in coda durante le ore notturne e quello durante le ore diurne, come era lecito aspettarsi visti i tassi di generazione variabili. È interessante notare come il più alto numero di auto in coda sia 8, con una situazione tende ad assestarsi ad un valore leggermente inferiore. È anche importante considerare che, nonostante i tempi di inter-arrivo notturni siano estremamente elevati, come specificato in precedenza, comunque durante la notte tendano ad accodarsi alcune macchine, fino a un massimo di 5.

Questa è una delle spiegazioni per cui nelle grandi città nelle ore meno affollate molti semafori vengono spenti. Come sarà chiaro più avanti, tuttavia, con un algoritmo di gestione dinamica dell’incrocio, il problema dell’accodamento viene risolto, seppur mantenendo acceso l’apparato semaforico, contribuendo alla sicurezza dei guidatori anche a tarda sera.

La \textit{figura \ref{fig:TempoDiAttesaStatico}} rappresenta invece il tempo medio di attesa delle auto nella corsia considerata, sempre in funzione dell’ora del giorno. Sebbene il tempo concesso a queste auto per defluire sia alquanto importante (ben 25 secondi), si nota come l’attesa tocchi un picco di quasi 95 secondi, in corrispondenza dell’ora di punta, per poi decrescere, seppur di poco. È altresì fondamentale considerare che, proprio per la mancanza di distinzione nel tempo concesso alla luce verde per ciascuna corsia, anche di notte, con un bassissimo affollamento, si assiste ad un’attesa media di più di 35 secondi.

Appare dunque chiaro come siano ampi i margini di miglioramento, sia per quanto riguarda il numero massimo di auto in coda, che, soprattutto, per quanto concerne i tempi medi di attesa, e nel secondo capitolo di questa trattazione saranno analizzati vari metodi per ottimizzare questi due parametri.
\begin{center}
	\begin{figure}[H]
  \includegraphics[width=0.90\textwidth]{AutoInCodaStatico.pdf}
  \caption{Numero di auto in coda per una singola corsia}
  \label{fig:AutoInCodaStatico}
\end{figure}
\begin{figure}[H]
  \includegraphics[width=0.90\textwidth]{TempoDiAttesaStatico.pdf}
  \caption{Tempi di attesa medi per una singola corsia in funzione dell'orario considerato}
  \label{fig:TempoDiAttesaStatico}
\end{figure}
\end{center}
\section{Tabella riassuntiva dei risultati con valori variabili del parametro $\mu$}

Nella tabella presente alla seguente pagina sono riportate più esecuzioni dell'algoritmo, sia a fronte dello stesso intervallo relativo al parametro $\mu$, sia per intervalli diversi, per riportare come variano il tempo medio di attesa, il numero di veicoli che attraversano l'incrocio, il numero massimo di veicoli in coda per ogni esecuzione. Tutti questi dati sono inerenti alla singola corsia analizzata, che cambia di esecuzione in esecuzione, come è facile comprendere. 

Nello specifico si può notare come nella prima simulazione, con tassi di generazione elevati (mediamente all'ora di pranzo il tempo di arrivo fra una macchina e l'altra è minore di 10 secondi) si accodi un numero di macchine esagerato nella corsia presa in considerazione. La simulazione è stata infatti portata al limite in questo modo, ed è ovvio che un congestionamento di questo genere è alquanto raro, sebbene non inverosimile; l'obiettivo è stato quello di rendere il lettore consapevole del legame fra il parametro $\mu$ e tutti gli altri valori relativi alla simulazione stessa. Nel capitolo successivo si confronteranno vari scenari, e sarà più immediato capire come, a parità di $\mu$, qualsiasi sia il suo valore, un algoritmo di gestione dinamica è sicuramente più performante.
Risulta anche importante considerare che, a fronte di un sensibile calo del numero di automobili che attraversano l'incrocio, se è vero che cala anche il numero massimo di auto in coda, i tempi di attesa medi restano comunque molto elevati. Sempre al di sopra dei 35 secondi, assolutamente spropositati rispetto a quelli ottenuti con l'algoritmo che verrà proposto nel capitolo successivo.


\begin{table}[H]
\centering
\begin{tabular}{|C{2cm}|c|C{2.25cm}|C{2.25cm}|C{2.25cm}|C{2.25cm}|}
  \hline
  \multicolumn{6}{|c|}{\large Prima simulazione} \\\hline\hline
  \multicolumn{2}{|c|}{\textbf{Intervallo}} &
  \multicolumn{2}{C{5cm}|}{\textbf{Stima tempo interarrivo con minimo $\mu$}} &
  \multicolumn{2}{C{5cm}|}{\textbf{Stima tempo interarrivo con massimo $\mu$}} \\\hline
  \multicolumn{2}{|c|}{$[23, 28]$} &
  \multicolumn{2}{c|}{15.94 secondi} &
  \multicolumn{2}{c|}{19.41 secondi} \\\hline
  \multicolumn{6}{|c|}{\textit{Corsia: Ovest - Dritto/Destra}} \\\hline
  \textbf{\footnotesize Esecuzione} & 
  \textbf{\footnotesize $\mu$} & 
  \textbf{\footnotesize Tempo di arrivo medio} & 
  \textbf{\footnotesize Numero di auto totale} & 
  \textbf{\footnotesize Max numero auto in coda} & 
  \textbf{\footnotesize Tempo di attesa medio} \\\hline
  1 & 
  23.4080 & 
  $16.25s$ & 
  3143 &
  39 &
  $112.61s$ \\\hline
  2 & 
  24.8866 & 
  $17.25s$ & 
  2888 &
  23 &
  $67.03s$ \\\hline
  3 & 
  26.2806 & 
  $18.22s$ & 
  2809 &
  21 &
  $60.08s$ \\\hline
  \multicolumn{6}{c}{}\\\hline
  \multicolumn{6}{|c|}{\large Seconda simulazione} \\\hline\hline
  \multicolumn{2}{|c|}{\textbf{Intervallo}} &
  \multicolumn{2}{C{5cm}|}{\textbf{Stima tempo interarrivo con minimo $\mu$}} &
  \multicolumn{2}{C{5cm}|}{\textbf{Stima tempo interarrivo con massimo $\mu$}} \\\hline
  \multicolumn{2}{|c|}{$[28, 33]$} &
  \multicolumn{2}{c|}{19.41 secondi} &
  \multicolumn{2}{c|}{22.87 secondi} \\\hline
  \multicolumn{6}{|c|}{\textit{Corsia: Nord - Dritto/Destra}} \\\hline
  \textbf{\footnotesize Esecuzione} & 
  \textbf{\footnotesize $\mu$} & 
  \textbf{\footnotesize Tempo di arrivo medio} & 
  \textbf{\footnotesize Numero di auto totale} & 
  \textbf{\footnotesize Max numero auto in coda} & 
  \textbf{\footnotesize Tempo di attesa medio} \\\hline
  1 & 
  32.0241 & 
  $22.20s$ & 
  2314 &
  12 &
  $39.54s$ \\\hline
  2 & 
  30.4146 & 
  $21.08s$ & 
  2381 &
  13 &
  $43.12s$ \\\hline
  3 & 
  28.5156 & 
  $19.77s$ & 
  2553 &
  13 &
  $45.11s$ \\\hline
  \multicolumn{6}{c}{}\\\hline
  \multicolumn{6}{|c|}{\large Terza simulazione} \\\hline\hline
  \multicolumn{2}{|c|}{\textbf{Intervallo}} &
  \multicolumn{2}{C{5cm}|}{\textbf{Stima tempo interarrivo con minimo $\mu$}} &
  \multicolumn{2}{C{5cm}|}{\textbf{Stima tempo interarrivo con massimo $\mu$}} \\\hline
  \multicolumn{2}{|c|}{$[33, 38]$} &
  \multicolumn{2}{c|}{22.87 secondi} &
  \multicolumn{2}{c|}{26.34 secondi} \\\hline
  \multicolumn{6}{|c|}{\textit{Corsia: Sud - Sinistra}} \\\hline
  \textbf{\footnotesize Esecuzione} & 
  \textbf{\footnotesize $\mu$} & 
  \textbf{\footnotesize Tempo di arrivo medio} & 
  \textbf{\footnotesize Numero di auto totale} & 
  \textbf{\footnotesize Max numero auto in coda} & 
  \textbf{\footnotesize Tempo di attesa medio} \\\hline
  1 & 
  37.3261 & 
  $25.87s$ & 
  1960 &
  9 &
  $35.55s$ \\\hline
  2 & 
  33.6499 & 
  $23.32s$ & 
  2206 &
  9 &
  $38.69s$ \\\hline
  3 & 
  36.4679 & 
  $25.28s$ & 
  1923 &
  7 &
  $35.61s$ \\\hline
\end{tabular}
\caption{Esecuzioni dell'algoritmo di gestione statica al variare di $\mu$}
\label{table:tabellaconfrontostatico}
\end{table}








%inizio capitolo 2
\chapter{Algoritmo di gestione dinamica applicato ad un singolo incrocio \label{capitolo2}}

In questo capitolo si vuole analizzare il comportamento dello stesso incrocio visto in precedenza con un algoritmo di gestione ottimizzata, che conti il numero di auto in ciascuna corsia e in base a questo e ad altri parametri conceda il verde per un numero di secondi variabile. È da precisare che l’algoritmo utilizzato si basa su quello proposto da \textit{Maram Bani Younes e Azzedine Boukerche}\cite{itlc}, tuttavia sono state effettuate alcune importanti modifiche, che verranno analizzate nel corso della trattazione. 

Si è previsto,  oltre che un raffronto fra l'algoritmo di gestione statica e quello sviluppato in autonomia, modificando quanto proposto dai due ingegneri, anche un meccanismo per valutare l'efficacia di queste modifiche, confrontando il nuovo codice con una implementazione di quello presente nella pubblicazione citata. Tutti i confronti, con i relativi grafici e le relative conclusioni, verranno presentati alla fine di questo capitolo, dopo aver descritto nel dettaglio tutti gli aspetti relativi alle modifiche apportate al modello e ai meccanismi implementativi adottati.


Per quanto riguarda lo pseudocodice progettato dai due ingegneri, è disponibile alla seguente pagina.
\newpage
\begin{lstlisting}[language=C,label=pseudocode,caption=Pseudocodice di gestione dinamica di un incrocio]
while d_i di una qualsiasi corsia > 0 do
{
	
	sia j la corsia con il maggior numero di macchine;
	siano i_1 e i_2 le due corsie complementari;
		if (d_i_1 > d_i_2)
		{
			schedula j,i_1;
			d_i_1 = 0;
			t_i_1 = 0;
		}
		
		else
		{
			schedula j,i_2;
			d_i_2 = 0;
			t_i_2 = 0;
		}
		d_j = 0;
		t_j = 0;
	
}   
\end{lstlisting}
\newpage
\section{Breve spiegazione dell'algoritmo originale}
La proposta di \textit{Maram Bani Younes e Azzedine Boukerche} si basa sulla definizione di alcuni parametri, il primo dei quali è la così detta \textit{Ready Area}. Tale area è sostanzialmente il range entro il quale le macchine si possono considerare in coda per l’incrocio: si individua, per ogni semaforo della giunzione, una distanza massima, di solito corrispondente alla portata del dispositivo di rilevazione in termini di veicoli, oltre la quale le vetture non vengono più conteggiate. 

Il dispositivo in questione può essere rappresentato da una semplice telecamera che, mediante meccanismi tipici della Computer Vision, utilizzando API quali Google Vision\cite{google_vision} o OpenCV\cite{open_CV}, riesca ad individuare la variabile principale su cui si basa l’algoritmo: il numero di auto in coda per ciascuna corsia. L’implementazione hardware di questo sistema, tuttavia, non è oggetto di questa tesi e dunque non ci si dilungherà oltre nella sua spiegazione.

Il tempo concesso alle corsie prescelte (durata della luce verde) è poi dinamico, e viene calcolato sulla base della seguente formula: 

\begin{equation} \label{eq:}
  T = \theta + \frac{F_d}{S_{tf}}
\end{equation}

Il parametro $\theta$ è una costante, e rappresenta il tempo che mediamente è necessario alla prima macchina per partire, mentre $F_d$ è la distanza del veicolo più lontano dal semaforo (sempre interno alla \textit{Ready Area} ). In ultimo, $S_{tf}$ è la velocità media del flusso di auto nell’intersezione. Come è ovvio notare, sia $\theta$ che $S_{tf}$ sono ottenuti mediante delle stime, non essendo misurabili prima della effettiva esecuzione dell’algoritmo. 

Una proposta alternativa, certamente più efficiente ma anche meno facile da implementare, è quella di permettere alle automobili di comunicare fra loro e con il semaforo dati quali la $F_d$ (distanza), nonché la loro velocità media. Certamente in questo modo si renderebbe l’esecuzione dello script più veritiera ed affidabile, ed in futuro, grazie agli sforzi delle case automobilistiche, probabilmente tutto ciò sarà possibile.

Altri parametri da considerare sono $d_i$ che rappresenta il numero di automobili in coda nella corsia i-esima e $t_i$, tempo richiesto a tutti i veicoli della corsia i-esima (all’interno della \textit{Ready Area}) per lasciare l’intersezione.

\textbf{Il funzionamento dell’algoritmo è dunque il seguente:} si individua la corsia più affollata fra quella che presentano almeno un’automobile, e contestualmente si rintracciano le due corsie compatibili con essa, così come spiegato nel \textit{Capitolo \ref{capitolo1}} e come si può notare nella \textit{figura \ref{dirCompatibili}}.

A questo punto, fra queste due candidate si sceglie ancora la più affollata; in questo modo non solo si va a concedere il verde alla strada con più automobili, ma si cerca anche di ottimizzare la scelta della sua complementare, che non resta statica (come invece è nell’algoritmo utilizzato nel capitolo precedente), ma diventa dinamica ed intelligente. Successivamente l’algoritmo prevede di calcolare il tempo \textit{T}, concedendolo alla luce verde delle due corsie, e di azzerare le variabili $d_i, d_j$ e $t_i, t_j$ relative, in quanto si prevede che dopo T secondi le corsie in questione presentino un numero di veicoli pari a zero.
\newpage
\section{Le modifiche apportate}
Sono molteplici le modifiche apportate all’algoritmo proposto, alcune delle quali a puro scopo implementativo, altre come migliorie.

In primo luogo, si è preferito non considerare una \textit{Ready Area} per non dover dare alle code di \textit{Simulink} un massimo numero di entità (veicoli) da contenere, questo per effettuare più simulazioni variando i tassi di generazione, verificando il comportamento del modello anche in condizioni esageratamente negative, con code anche molto affollate (più di 50 automobili). 

Il concetto della \textit{Ready Area}, tuttavia, non è stato abbandonato del tutto, ma è stato piuttosto migliorato: si è deciso di concedere un tempo minimo ed un tempo massimo alla durata del verde per ciascuna corsia. Il tempo minimo serve perché, anche con una sola macchina in coda, si è voluto stabilizzare la situazione, che altrimenti si sarebbe tradotta in un passaggio troppo repentino dal verde al rosso. Per quanto concerne il tempo massimo, questo è diretta espressione della \textit{Ready Area}: invece che considerare un massimo numero di veicoli, si considera che, indipendentemente da questo numero, il verde non può durare più di un certo valore, questo per evitare tempi di attesa troppo lunghi nelle corsie meno affollate, ma comunque in cui sono presenti veicoli. 

Il tempo concesso alla luce verde, poi, viene calcolato tenendo conto della distribuzione uniforme che governa gli \textit{Entity Server}, precedentemente descritta, assegnando ad ogni macchina un valore pari al valore medio della distribuzione (in questa simulazione pari a $5s$), ed ottenendo il tempo complessivo moltiplicando questo valore per il numero di veicoli in coda. A questo punto, se il numero ottenuto è inferiore al tempo minimo, esso si scarta, e si prende in considerazione proprio quel minimo, viceversa se supera il tempo massimo è questo ad essere tenuto in considerazione e ad essere concesso. Pertanto, poiché non è detto che le code a cui viene dato il via libera si svuotino, non si va ad azzerare i contatori relativi, ma semplicemente all'esecuzione successiva della funzione questi sono aggiornati con i nuovi dati rilevati, ottenuti dalle code stesse.

Inoltre, è stata individuata una grande lacuna nell’algoritmo oggetto della pubblicazione citata: l’assenza di un meccanismo di gestione della \textit{starvation}. 

Considerando esclusivamente il \textit{codice \ref{pseudocode}}, infatti, si nota che se in una strada continuano a confluire veicoli più velocemente che nelle altre, a questa sarà sempre concesso il verde. Questo comportamento da un lato è positivo perché si va a decongestionare tale corsia, ma è ovviamente improponibile, in quanto potrebbe paradossalmente capitare che un semaforo resti rosso per ore ed ore, e che una persona sia costretta ad attendere un tempo esagerato prima di passare, perché la sua strada è meno affollata.

È stato scelto dunque di implementare un meccanismo di gestione della \textit{starvation} mediante un algoritmo \textit{LRU} (Least Recently Used) opportunamente modificato. Ad ogni corsia è associato un contatore, ed ogni qual volta ad una strada viene concesso il verde, il suo contatore viene azzerato, mentre quello delle altre si incrementa del numero di secondi per cui è stato concesso il verde alla corsia suddetta. In questo modo, quando un contatore raggiunge un valore limite (liberamente impostabile), la relativa strada ottiene priorità massima, e l’algoritmo viene applicato in maniera identica (calcolo del tempo del verde e selezione della complementare) come se essa fosse quella più affollata. In particolare il tempo massimo di attesa prima che scatti la \textit{starvation} è stato impostato a 2.5 minuti.

È anche da notare che sono stati apportati alcuni minimi cambiamenti al modello: le code sono state collegate agli ingressi del blocco contenente la funzione da eseguire, comunicando ad essa il numero di veicoli presenti, inserendo delle \textbf{Input Port \cite{inputport}} nel blocco Function Call Subsystem, ed il canale di \textit{feedback} è stato utilizzato per implementare il meccanismo della gestione della \textit{starvation}, e non per conoscere a quale semaforo è stato concesso il verde nell’esecuzione precedente, informazione ora inutile. Non è stato variato altro, se non, ovviamente, il codice nel blocco \textit{Matlab Function}, quindi non si ravvisa la necessità di descrivere nei dettagli nuovamente l'intero modello, il cui schema è comunque presente alla pagina seguente, seguito dal codice che lo governa.
\newline
\begin{figure}[H]
\centering
  \includegraphics[width=0.25\textwidth]{InputPort.pdf}
  \caption{Input Port}
  \label{fig:inputport}
\end{figure}

\begin{figure}[H]
  \includegraphics[width=1\textwidth]{SingoloIncrocioAlgoritmo.pdf}
  \caption{Modello di un incrocio a raso a quattro bracci in Simulink e SimEvents, gestione dinamica}
  \label{fig:modellosingoloincrocioalgoritmo}
\end{figure}
\newpage

\begin{lstlisting}[language=Matlab,label=algoritmodin,caption=Implementazione dell'algoritmo di gestione dinamica di un singolo incrocio]
function [delay, y1, y2, y3, y4, y5, y6, y7, y8, feedback] = fcn(u1, u2, u3, u4, u5, u6, u7, u8, history)


% VARIABILI DI USCITA
% Prima attesa, prima dell'esecuzione dell'algoritmo la prima volta
delay = 1; 
y1 = 0; 
y2 = 0; 
y3 = 0; 
y4 = 0; 
y5 = 0; 
y6 = 0; 
y7 = 0;
y8 = 0;
feedback = zeros([4 2]);
 
% Numero di veicoli per ogni direzione
M = ([u1 u2; u3 u4; u5 u6; u7 u8;]);
 
% Tempo medio per fluire per ogni macchina
Tv = 5;
 
% Massimo tempo del verde
MaxGreen = 35;
 
% Minimo tempo del verde
MinGreen = 15;
 
% Controllo sulla starvation (LRU). Tempo di attesa massimo: 2.5 minuti
starvation = 150;
 
% Cerco il massimo
[R, C] = size(M);
iMax = 1;
jMax = 1;
vMax = M(1, 1);
 
%Usato per la starvation, interrompe la ricerca
flag = 0;
 
for i = 1 : 1 : R
    for j = 1 : 1 : C
        
        if M(i, j) > 0 && history(i, j) > starvation
            vMax = M(i, j);
            iMax = i;
            jMax = j;
            flag = 1;
            break;
        end
 
        if M(i, j) > vMax
            vMax = M(i, j);
            iMax = i;
            jMax = j;
        end
       
    end
    
    if flag == 1
        break;
    end
end
 
 
% Cerco i possibili candidati
[c1_i, c1_j, c2_i, c2_j] = getCandidates(iMax, jMax);
 
%Scelgo tra i candidati
if M(c1_i, c1_j) > M(c2_i, c2_j) 
    iCan = c1_i;
    jCan = c1_j;
else
    iCan = c2_i;
    jCan = c2_j;
end
 
delay = vMax*Tv;
 
if vMax*Tv > MaxGreen
        delay = MaxGreen;
elseif vMax*Tv < MinGreen
        delay = MinGreen;
end
    
%Risultato
U = ([0 0; 0 0; 0 0; 0 0]);
 
U(iMax, jMax) = 1;
U(iCan, jCan) = 1;
     
history( : ) = history( : ) + delay;
    
history(iMax, jMax) = 0;
history(iCan, jCan) = 0;

%Output 
y1 = U(1, 1); y2 = U(1, 2); y3 = U(2, 1); y4 = U(2, 2); y5 = U(3, 1); y6 = U(3, 2); y7 = U(4, 1); y8 = U(4, 2); feedback(:) = history(:);
end
    
\end{lstlisting}
\newpage
\section{Spiegazione del codice implementato}
In primo luogo vengono inizializzate le variabili in uscita della funzione, corrispondenti al valore di trigger degli otto entity gate rappresentanti i vari semafori. Di default questi valori sono posti a 0 (semaforo rosso). In input si accetta il numero di veicoli presenti in ogni coda nell'istante in cui la funzione viene richiamata, nonché il vettore \textit{history}, usato per gestire la \textit{starvation}, il cui significato sarà chiaro più avanti.

I due \textit{cicli for} annidati servono ad individuare la strada con il numero di veicoli più elevato, purché non si verifichi il caso di \textit{starvation}. Tale strada (individuata da due indici, \textit{i} e \textit{j} all’interno della matrice \textit{M}, matrice che organizza gli otto input indipendenti), viene poi data in pasto alla funzione \textit{getCandidates}, il cui codice è alla pagina seguente e che restituisce gli indici relativi alla matrice M che individuano le due strade che possono essere scelte assieme alla principale. A questo punto fra le due corsie complementari candidate viene scelta, come già detto, quella più affollata. Viene calcolato il tempo da concedere al verde in funzione del numero di veicoli della strada scelta come principale, e vengono effettuati i controlli con il tempo minimo ed il tempo massimo, precedentemente dichiarati come costanti.

Successivamente, ogni elemento del vettore \textit{History}, il quale contiene i contatori inerenti all’algoritmo \textit{LRU} per ogni corsia, viene incrementato del tempo stabilito, azzerando poi gli elementi corrispondenti alle corsie a cui si sta concedendo il verde.

In output viene inviato \textit{1} agli \textit{Entity Gate} che devono aprirsi, corrispondenti alle strade individuate, e \textit{0} a tutti gli altri. 

Oltre a queste informazioni rappresenta un output anche il \textit{delay}, ovvero proprio il tempo concesso, che viene inviato al timer descritto nel \textit{capitolo \ref{capitolo1}}, che richiamerà la funzione allo scadere dello stesso.

La matrice M è stata creata seguendo la struttura della seguente tabella, ritenuta agevole per l'implementazione dell'algoritmo.
\newline
\begin{table}[H]
\centering
\begin{tabular}{|c|c|c|}
  \hline
  & 
  \textbf{Colonna 1} & 
  \textbf{Colonna 2} \\\hline
  Riga 1 & 
  Corsia Ovest - Sinistra & 
  Corsia Ovest - Dritto/Destra \\\hline
  Riga 2 & 
  Corsia Nord - Sinistra & 
  Corsia Nord - Dritto/Destra \\\hline
  Riga 3 & 
  Corsia Est - Sinistra & 
  Corsia Est - Dritto/Destra \\\hline
  Riga 4 & 
  Corsia Sud - Sinistra & 
  Corsia Sud - Dritto/Destra \\\hline
\end{tabular}
\caption{Matrice della lunghezza delle code per ogni corsia}
\label{table:matrix-length-corsia}
\end{table}

\newpage
\begin{lstlisting}[language=Matlab,label=key,caption=Scelta delle due strade complementari a quella selezionata come principale]
    function [i1, j1, i2, j2] = getCandidates(i, j)
    % Direzioni accoppiate (8x4)
    % Per ciascuna direzione scelta, ci sono 2 possibili direzioni
    % complementari
    % D = {[1 2; 3 1] [1 1; 3 2]; [4 1; 2 2] [2 1; 4 2]; [3 2; 1 1] [3 1; 1 2]; [4 2; 2 1] [4 1; 2 2]};
    
    i1 = 0;
    i2 = 0;
    j1 = 0;
    j2 = 0;
    
    if i==1 && j ==1
        i1 = 1;
        j1 = 2;
        i2 = 3;
        j2 = 1;
    elseif i==1 && j==2
       i1 = 1;
       j1 = 1;
       i2 = 3;
       j2 = 2;
    elseif i==2 && j==1
       i1 = 4;
       j1 = 1;
       i2 = 2;
       j2 = 2;
    elseif i==2 && j==2
       i1 = 2;
       j1 = 1;
       i2 = 4;
       j2 = 2;
    elseif i==3 && j==1
       i1 = 3;
       j1 = 2;
       i2 = 1;
       j2 = 1;
    elseif i==3 && j==2
       i1 = 3;
       j1 = 1;
       i2 = 1;
       j2 = 2;
    elseif i==4 && j==1
       i1 = 4;
       j1 = 2;
       i2 = 2;
       j2 = 1;
    elseif i==4 && j==2
       i1 = 4;
       j1 = 1;
       i2 = 2;
       j2 = 2;
    end
            
        
end
\end{lstlisting}

È da sottolineare che questo codice funziona fintanto che gli ingressi non vengono variati: se si dovesse collegare una strada ad un ingresso diverso da quello previsto ed utilizzato nella simulazione chiaramente l’intero algoritmo non sarebbe più affidabile. L'autore di questa tesi è consapevole che esistono metodi più eleganti ed ottimizzati rispetto a questa sequenza di \textit{if} ed \textit{elseif} per ottenere il medesimo scopo, tuttavia si è reso necessario procedere in questo modo per lo scarso supporto del linguaggio \textit{Matlab} a tali meccanismi, come, per esempio, gli \textit{switch - case}. In questo modo, inoltre, il codice risulta particolarmente leggibile e facile da comprendere.
\newline

Le differenze con un normale algoritmo di gestione di una giunzione, come quello proposto nel \textit{capitolo \ref{capitolo1}} sono quindi evidenti: in primo luogo non si segue una turnazione statica per quanto riguarda la concessione del verde, seguendo invece un meccanismo di priorità: ci si chiede infatti quale sia la strada che necessita del verde in quell'istante, più delle altre, e glielo si concede. Inoltre anche il tempo per cui questo verde è concesso è variabile: si ipotizzi per esempio di analizzare l'incrocio durante le ore notturne, non avrebbe senso concedere il verde per un numero di secondi elevato quando in coda è presente una sola macchina: si tratterebbe di una perdita di tempo, tradotta in un maggior consumo di carburante ed un maggiore stress per gli automobilisti delle altre corsie, costretti ad attendere perché viene concesso il verde ad una strada praticamente vuota. Viceversa è ovvio che durante il giorno, per una strada affollata, la durata della luce verde deve essere sufficiente da garantire che un buon numero delle macchine in coda defluisca.
\newpage

\section{Il modello per confrontare le diverse gestioni del singolo incrocio}

I benefici esposti, ovviamente, devono trovare riscontro in un confronto imparziale ed assolutamente valido fra i modelli presentati, affinché a partire dai medesimi dati in input (le automobili generate), e con gli stessi parametri (tasso di generazione, tempo di servizio interno agli \textit{Entity Server}) essi gestiscano l'incrocio secondo la propria logica. Si è scelto, pertanto, di creare in un nuovo workspace il modello identificato dalla \textit{figura \ref{fig:gatecontrolcompare}}. Si può notare come al suo interno siano presenti tre blocchi principali, ognuno dei quali è rappresentativo di un algoritmo di gestione. Guardando infatti ai tre blocchi presenti, si comprende subito come questi siano in realtà esattamente i modelli precedentemente descritti. Il primo, quello più in alto a sinistra, è quello presentato nel \textit{capitolo \ref{capitolo1}}, relativo ad una gestione statica (\textit{figura \ref{fig: modelloSingoloIncrocio}}). Quello in basso a sinistra invece si riferisce al modello di controllo dinamico, di cui si è parlato in questo capitolo (\textit{figura \ref{fig:modellosingoloincrocioalgoritmo}}), mentre quello in basso a destra è perfettamente identico a quest'ultimo, eccezion fatta per la gestione della \textit{starvation}, che si è eliminata. Ciò è stato fatto per capire come tale gestione, che comunque resta assolutamente necessaria, interferisca con il normale funzionamento dell'algoritmo stesso.

Nello specifico, in questo terzo modello, si è provveduto semplicemente a commentare la seguente porzione di codice, lasciando tutto il resto invariato.
\newline
\begin{lstlisting}[language=Matlab,label=starvation,caption= Porzione di codice relativa alla gestione della starvation]
    if M(i, j) > 0 && history(i, j) > starvation
            vMax = M(i, j);
            iMax = i;
            jMax = j;
            flag = 1;
            break;
    end
        ...
        ...
        ...
   if flag == 1
        break;
   end
\end{lstlisting}

Gli \textit{Entity Generator}, poi, sono stati collegati a dei blocchi di tipo \textbf{Entity Replicator\cite{entityreplicator}}, in modo tale che appena una entità viene generata in una corsia specifica, questa venga subito inviata ai tre modelli contemporaneamente. In questo modo i tre incroci sono esattamente identici sia per quanto riguarda il numero di entità che devono gestire che per quanto concerne quando queste si presentano, tutto questo per garantire l'assoluta imparzialità delle condizioni, precedentemente citata. Il Replicator è stato configurato seguendo lo schema in figura.
\newline

\begin{figure}[H]
\centering
  \includegraphics[width=1\textwidth]{replicatorEProprieta.pdf}
  \caption{Entity Replicator e relative proprietà}
  \label{fig:}
\end{figure}

Dunque il workflow è il seguente: quando una entità viene generata, questa viene inviata a tutti e tre gli incroci, in corrispondenza della stessa corsia e nello stesso istante di tempo. La simulazione parte allo stesso istante per tutti e tutti sono configurati secondo gli stessi parametri. Al termine della simulazione tutti gli incroci producono tre grafici per ogni corsia, relativi al tempo di attesa medio, al numero di macchine totali che hanno attraversato tale strada ed al numero di macchine in coda in funzione dell'ora del giorno. Confrontando i grafici di una stessa strada, e facendo questo per ogni strada presente, si ottiene una chiara idea di quale dei tre algoritmi stia funzionando meglio.

\newpage

\begin{figure}[H]
  \includegraphics[width=1\textwidth]{GateControlCompare.pdf}
  \caption{Modello per comparare gli algoritmi utilizzati per la gestione di un singolo incrocio}
  \label{fig:gatecontrolcompare}
\end{figure}

\newpage
\section{I risultati ottenuti}
Per le seguenti simulazioni il primo scenario, ovvero il seguente, presenta dei tassi di generazione randomici, esattamente come nel \textit{capitolo \ref{capitolo1}}, con il parametro $\mu$ scelto casualmente, secondo una distribuzione uniforme, fra 23 e 35 ed un tempo di servizio pari a $5s$, rispetto ad un numero randomico fra $4s$ e $6s$ come spiegato in precedenza, per eliminare questa ulteriore variabile casuale. Questo per ottenere un numero di automobili sì variabile, ma che sia confrontabile fra le varie strade, con una configurazione dunque che può dirsi bilanciata. Il codice interno agli \textit{Entity Generator} non è dunque stato variato rispetto a quanto presentato nel \textit{codice \ref{mu-definition}}, così come la definizione dei parametri globali (\textit{codice \ref{entity-generator}}).

Il tempo di concessione massima del verde, poi, è stato impostato a $35s$, e quello minimo a $15s$, come si può notare nel codice presentato precedentemente.

In una configurazione di questo genere sono stati ottenuti i risultati riportati nelle seguenti tabelle.
\newline

\begin{table}[H]
\centering
\begin{tabular}{|C{2cm}|C{1.65cm}|C{1.65cm}|C{1.65cm}|C{1.65cm}|C{1.65cm}|C{1.65cm}|}
  \hline
  \multicolumn{7}{|c|}{\large Prima simulazione: prime quattro corsie} \\\hline\hline
  \textbf{Intervallo} &
  \multicolumn{3}{C{5.8cm}|}{\textbf{Stima tempo interarrivo con minimo $\mu$}} &
  \multicolumn{3}{C{5.8cm}|}{\textbf{Stima tempo interarrivo con massimo $\mu$}} \\\hline
  $[23, 35]$ &
  \multicolumn{3}{c|}{$15.94s$} &
  \multicolumn{3}{c|}{$24.26$} \\\hline
  \multirow{7}{2cm}{\centering \textit{\footnotesize{Corsia Ovest \newline --- \newline Sinistra}}}
  & \multicolumn{2}{C{3.7cm}|}{\textbf{$\mu$}} 
  & \multicolumn{2}{C{3.7cm}|}{\textbf{\footnotesize{Tempo di arrivo medio}}} 
  & \multicolumn{2}{C{3.7cm}|}{\textbf{\footnotesize{Numero di auto totale}}} \\\cline{2-7}
  & \multicolumn{2}{C{3.7cm}|}{24.1705} 
  & \multicolumn{2}{C{3.7cm}|}{$16.75s$} 
  & \multicolumn{2}{C{3.7cm}|}{2940} \\\cline{2-7}
  & \multicolumn{2}{C{3.7cm}|}{\footnotesize{\textbf{Gestione statica}}} 
  & \multicolumn{2}{C{3.7cm}|}{\footnotesize{\textbf{Gestione dinamica senza starvation}}} 
  & \multicolumn{2}{C{3.7cm}|}{\footnotesize{\textbf{Gestione dinamica con starvation}}} \\\cline{2-7}
  & \scriptsize{Max numero auto in coda}
  & \scriptsize{Tempo di attesa medio}
  & \scriptsize{Max numero auto in coda}
  & \scriptsize{Tempo di attesa medio}
  & \scriptsize{Max numero auto in coda}
  & \scriptsize{Tempo di attesa medio}\\\cline{2-7}
  & 25
  & $59.44s$
  & 11
  & $19.18s$
  & 13
  & $19.07s$\\\hline\hline
    \multirow{7}{2cm}{\centering \textit{\footnotesize{Corsia Ovest \newline --- \newline Dritto/Destra}}}
  & \multicolumn{2}{C{3.7cm}|}{\textbf{$\mu$}} 
  & \multicolumn{2}{C{3.7cm}|}{\textbf{\footnotesize{Tempo di arrivo medio}}} 
  & \multicolumn{2}{C{3.7cm}|}{\textbf{\footnotesize{Numero di auto totale}}} \\\cline{2-7}
  & \multicolumn{2}{C{3.7cm}|}{26.3420} 
  & \multicolumn{2}{C{3.7cm}|}{$18.26s$} 
  & \multicolumn{2}{C{3.7cm}|}{2809} \\\cline{2-7}
  & \multicolumn{2}{C{3.7cm}|}{\footnotesize{\textbf{Gestione statica}}} 
  & \multicolumn{2}{C{3.7cm}|}{\footnotesize{\textbf{Gestione dinamica senza starvation}}} 
  & \multicolumn{2}{C{3.7cm}|}{\footnotesize{\textbf{Gestione dinamica con starvation}}} \\\cline{2-7}
  & \scriptsize{Max numero auto in coda}
  & \scriptsize{Tempo di attesa medio}
  & \scriptsize{Max numero auto in coda}
  & \scriptsize{Tempo di attesa medio}
  & \scriptsize{Max numero auto in coda}
  & \scriptsize{Tempo di attesa medio}\\\cline{2-7}
  & 18
  & $52.61s$
  & 8
  & $17.27s$
  & 10
  & $17.34s$\\\hline\hline
  \multirow{7}{2cm}{\centering \textit{\footnotesize{Corsia Nord \newline --- \newline Sinistra}}}
  & \multicolumn{2}{C{3.7cm}|}{\textbf{$\mu$}} 
  & \multicolumn{2}{C{3.7cm}|}{\textbf{\footnotesize{Tempo di arrivo medio}}} 
  & \multicolumn{2}{C{3.7cm}|}{\textbf{\footnotesize{Numero di auto totale}}} \\\cline{2-7}
  & \multicolumn{2}{C{3.7cm}|}{29.5626} 
  & \multicolumn{2}{C{3.7cm}|}{$20.49s$} 
  & \multicolumn{2}{C{3.7cm}|}{2384} \\\cline{2-7}
  & \multicolumn{2}{C{3.7cm}|}{\footnotesize{\textbf{Gestione statica}}} 
  & \multicolumn{2}{C{3.7cm}|}{\footnotesize{\textbf{Gestione dinamica senza starvation}}} 
  & \multicolumn{2}{C{3.7cm}|}{\footnotesize{\textbf{Gestione dinamica con starvation}}} \\\cline{2-7}
  & \scriptsize{Max numero auto in coda}
  & \scriptsize{Tempo di attesa medio}
  & \scriptsize{Max numero auto in coda}
  & \scriptsize{Tempo di attesa medio}
  & \scriptsize{Max numero auto in coda}
  & \scriptsize{Tempo di attesa medio}\\\cline{2-7}
  & 10
  & $38.86s$
  & 9
  & $24.02s$
  & 10
  & $22.85s$\\\hline\hline
  \multirow{7}{2cm}{\centering \textit{\footnotesize{Corsia Nord \newline --- \newline Dritto/Destra}}}
  & \multicolumn{2}{C{3.7cm}|}{\textbf{$\mu$}} 
  & \multicolumn{2}{C{3.7cm}|}{\textbf{\footnotesize{Tempo di arrivo medio}}} 
  & \multicolumn{2}{C{3.7cm}|}{\textbf{\footnotesize{Numero di auto totale}}} \\\cline{2-7}
  & \multicolumn{2}{C{3.7cm}|}{34.4901} 
  & \multicolumn{2}{C{3.7cm}|}{$23.91s$} 
  & \multicolumn{2}{C{3.7cm}|}{2049} \\\cline{2-7}
  & \multicolumn{2}{C{3.7cm}|}{\footnotesize{\textbf{Gestione statica}}} 
  & \multicolumn{2}{C{3.7cm}|}{\footnotesize{\textbf{Gestione dinamica senza starvation}}} 
  & \multicolumn{2}{C{3.7cm}|}{\footnotesize{\textbf{Gestione dinamica con starvation}}} \\\cline{2-7}
  & \scriptsize{Max numero auto in coda}
  & \scriptsize{Tempo di attesa medio}
  & \scriptsize{Max numero auto in coda}
  & \scriptsize{Tempo di attesa medio}
  & \scriptsize{Max numero auto in coda}
  & \scriptsize{Tempo di attesa medio}\\\cline{2-7}
  & 11
  & $37.12s$
  & 8
  & $21.31s$
  & 7
  & $21.43s$\\\hline
\end{tabular}
\caption{Tabella di comparazione fra algoritmi di gestione del singolo incrocio - prime quattro corsie - $\mu$ casuali}
\label{table:keytable}
\end{table}
\newpage

\begin{table}[H]
\centering
\begin{tabular}{|C{2cm}|C{1.65cm}|C{1.65cm}|C{1.65cm}|C{1.65cm}|C{1.65cm}|C{1.65cm}|}
  \hline
  \multicolumn{7}{|c|}{\large Prima simulazione: ultime quattro corsie} \\\hline\hline
  \textbf{Intervallo} &
  \multicolumn{3}{C{5.8cm}|}{\textbf{Stima tempo interarrivo con minimo $\mu$}} &
  \multicolumn{3}{C{5.8cm}|}{\textbf{Stima tempo interarrivo con massimo $\mu$}} \\\hline
  $[23, 35]$ &
  \multicolumn{3}{c|}{$15.94s$} &
  \multicolumn{3}{c|}{$24.26$} \\\hline
  \multirow{7}{2cm}{\centering \textit{\footnotesize{Corsia Est \newline --- \newline Sinistra}}}
  & \multicolumn{2}{C{3.7cm}|}{\textbf{$\mu$}} 
  & \multicolumn{2}{C{3.7cm}|}{\textbf{\footnotesize{Tempo di arrivo medio}}} 
  & \multicolumn{2}{C{3.7cm}|}{\textbf{\footnotesize{Numero di auto totale}}} \\\cline{2-7}
  & \multicolumn{2}{C{3.7cm}|}{34.5787} 
  & \multicolumn{2}{C{3.7cm}|}{$23.97s$} 
  & \multicolumn{2}{C{3.7cm}|}{2152} \\\cline{2-7}
  & \multicolumn{2}{C{3.7cm}|}{\footnotesize{\textbf{Gestione statica}}} 
  & \multicolumn{2}{C{3.7cm}|}{\footnotesize{\textbf{Gestione dinamica senza starvation}}} 
  & \multicolumn{2}{C{3.7cm}|}{\footnotesize{\textbf{Gestione dinamica con starvation}}} \\\cline{2-7}
  & \scriptsize{Max numero auto in coda}
  & \scriptsize{Tempo di attesa medio}
  & \scriptsize{Max numero auto in coda}
  & \scriptsize{Tempo di attesa medio}
  & \scriptsize{Max numero auto in coda}
  & \scriptsize{Tempo di attesa medio}\\\cline{2-7}
  & 9
  & $36.93s$
  & 7
  & $18.76s$
  & 8
  & $19.08s$\\\hline\hline
    \multirow{7}{2cm}{\centering \textit{\footnotesize{Corsia Est \newline --- \newline Dritto/Destra}}}
  & \multicolumn{2}{C{3.7cm}|}{\textbf{$\mu$}} 
  & \multicolumn{2}{C{3.7cm}|}{\textbf{\footnotesize{Tempo di arrivo medio}}} 
  & \multicolumn{2}{C{3.7cm}|}{\textbf{\footnotesize{Numero di auto totale}}} \\\cline{2-7}
  & \multicolumn{2}{C{3.7cm}|}{24.8914} 
  & \multicolumn{2}{C{3.7cm}|}{$17.25s$} 
  & \multicolumn{2}{C{3.7cm}|}{2928} \\\cline{2-7}
  & \multicolumn{2}{C{3.7cm}|}{\footnotesize{\textbf{Gestione statica}}} 
  & \multicolumn{2}{C{3.7cm}|}{\footnotesize{\textbf{Gestione dinamica senza starvation}}} 
  & \multicolumn{2}{C{3.7cm}|}{\footnotesize{\textbf{Gestione dinamica con starvation}}} \\\cline{2-7}
  & \scriptsize{Max numero auto in coda}
  & \scriptsize{Tempo di attesa medio}
  & \scriptsize{Max numero auto in coda}
  & \scriptsize{Tempo di attesa medio}
  & \scriptsize{Max numero auto in coda}
  & \scriptsize{Tempo di attesa medio}\\\cline{2-7}
  & 32
  & $65.24s$
  & 11
  & $22.33s$
  & 11
  & $22.48s$\\\hline\hline
  \multirow{7}{2cm}{\centering \textit{\footnotesize{Corsia Sud \newline --- \newline Sinistra}}}
  & \multicolumn{2}{C{3.7cm}|}{\textbf{$\mu$}} 
  & \multicolumn{2}{C{3.7cm}|}{\textbf{\footnotesize{Tempo di arrivo medio}}} 
  & \multicolumn{2}{C{3.7cm}|}{\textbf{\footnotesize{Numero di auto totale}}} \\\cline{2-7}
  & \multicolumn{2}{C{3.7cm}|}{34.6471} 
  & \multicolumn{2}{C{3.7cm}|}{$20.02s$} 
  & \multicolumn{2}{C{3.7cm}|}{1996} \\\cline{2-7}
  & \multicolumn{2}{C{3.7cm}|}{\footnotesize{\textbf{Gestione statica}}} 
  & \multicolumn{2}{C{3.7cm}|}{\footnotesize{\textbf{Gestione dinamica senza starvation}}} 
  & \multicolumn{2}{C{3.7cm}|}{\footnotesize{\textbf{Gestione dinamica con starvation}}} \\\cline{2-7}
  & \scriptsize{Max numero auto in coda}
  & \scriptsize{Tempo di attesa medio}
  & \scriptsize{Max numero auto in coda}
  & \scriptsize{Tempo di attesa medio}
  & \scriptsize{Max numero auto in coda}
  & \scriptsize{Tempo di attesa medio}\\\cline{2-7}
  & 11
  & $37.92s$
  & 8
  & $27.96s$
  & 10
  & $27.16s$\\\hline\hline
  \multirow{7}{2cm}{\centering \textit{\footnotesize{Corsia Sud \newline --- \newline Dritto/Destra}}}
  & \multicolumn{2}{C{3.7cm}|}{\textbf{$\mu$}} 
  & \multicolumn{2}{C{3.7cm}|}{\textbf{\footnotesize{Tempo di arrivo medio}}} 
  & \multicolumn{2}{C{3.7cm}|}{\textbf{\footnotesize{Numero di auto totale}}} \\\cline{2-7}
  & \multicolumn{2}{C{3.7cm}|}{34.4860} 
  & \multicolumn{2}{C{3.7cm}|}{$23.90s$} 
  & \multicolumn{2}{C{3.7cm}|}{2175} \\\cline{2-7}
  & \multicolumn{2}{C{3.7cm}|}{\footnotesize{\textbf{Gestione statica}}} 
  & \multicolumn{2}{C{3.7cm}|}{\footnotesize{\textbf{Gestione dinamica senza starvation}}} 
  & \multicolumn{2}{C{3.7cm}|}{\footnotesize{\textbf{Gestione dinamica con starvation}}} \\\cline{2-7}
  & \scriptsize{Max numero auto in coda}
  & \scriptsize{Tempo di attesa medio}
  & \scriptsize{Max numero auto in coda}
  & \scriptsize{Tempo di attesa medio}
  & \scriptsize{Max numero auto in coda}
  & \scriptsize{Tempo di attesa medio}\\\cline{2-7}
  & 10
  & $36.63s$
  & 11
  & $29.44s$
  & 8
  & $27.69s$\\\hline
\end{tabular}
\caption{Tabella di comparazione fra algoritmi di gestione del singolo incrocio - ultime quattro corsie - $\mu$ casuali}
\label{table:keytable}
\end{table}



Sono riportati, inoltre, alcuni grafici indicativi, inerenti ad una singola corsia (sempre la stessa, quella del braccio ovest relativa alla svolta a sinistra), per ciascuno dei tre modelli, che mostrano il tempo medio d'attesa ed il numero di macchine in coda in funzione dell'ora del giorno.
\newline

\textbf{Nota:} sui grafici relativi ai tempi di attesa viene riportata un'attesa stimata secondo una media temporale cumulativa, ovvero istante per istante sono sommati i tempi di attesa di ogni automobile che ha raggiunto l'incrocio attraverso una strada specifica e sono divisi per il numero complessivo di auto della corsia in questione. L'attesa relativa alla gestione della \textit{starvation}, invece, viene calcolata in base al tempo trascorso fra due concessioni consecutive del verde allo stesso semaforo. Risulta chiaro quindi come in realtà la \textit{starvation} intervenga anche se nei grafici che seguono il tempo di attesa medio non raggiunge mai i 2.5 minuti, in quanto sono due valori diversi. Ci si può rendere conto del fatto che sia intervenuta la \textit{starvation} nella configurazione relativa dal fatto che i valori ottenuti si discostino da quelli inerenti alla gestione dinamica senza \textit{starvation}, seppur di poco. Essendo infatti gli input ed i parametri dei due modelli identici, se la \textit{starvation} non fosse intervenuta, anche gli output sarebbero dovuti essere uguali.

\begin{figure}[H]
\centering
  \includegraphics[width=1\textwidth]{figuraPartiti.pdf}
  \caption{Auto che hanno attraversato la corsia Ovest - Sinistra (uguale per i tre modelli di gestione)}
  \label{fig:partitiMuRandom}
\end{figure}

\begin{figure}[H]
\centering
  \includegraphics[width=0.9\textwidth]{figuraCodaGestioneStatica.pdf}
  \caption{Numero di auto in coda in funzione dell'ora del giorno - Gestione statica del singolo incrocio - $\mu$ casuali - corsia Ovest - Sinistra}
  \label{fig:}
\end{figure}

\begin{figure}[H]
\centering
  \includegraphics[width=0.9\textwidth]{figuraAttesaGestioneStatica.pdf}
  \caption{Tempi di attesa in funzione dell'ora del giorno - Gestione statica del singolo incrocio - $\mu$ casuali - corsia Ovest - Sinistra}
  \label{fig:}
\end{figure}
\newpage

\begin{figure}[H]
\centering
  \includegraphics[width=0.9\textwidth]{figuraCodaGestioneDinamica.pdf}
  \caption{Numero di auto in coda in funzione dell'ora del giorno - Gestione dinamica del singolo incrocio senza starvation - $\mu$ casuali - corsia Ovest - Sinistra}
  \label{fig:}
\end{figure}
\begin{figure}[H]
\centering
  \includegraphics[width=0.9\textwidth]{figuraAttesaGestioneDinamica.pdf}
  \caption{Tempi di attesa in funzione dell'ora del giorno - Gestione dinamica del singolo incrocio senza starvation - $\mu$ casuali - corsia Ovest - Sinistra}
  \label{fig:}
\end{figure}
\newpage
\begin{figure}[H]
\centering
  \includegraphics[width=0.9\textwidth]{figuraCodaGestioneDinamicaStarvation.pdf}
  \caption{Numero di auto in coda in funzione dell'ora del giorno - Gestione dinamica del singolo incrocio con starvation - $\mu$ casuali - corsia Ovest - Sinistra}
  \label{fig:}
\end{figure}
\begin{figure}[H]
\centering
  \includegraphics[width=0.9\textwidth]{figuraAttesaGestioneDinamicaStarvation.pdf}
  \caption{Tempi di attesa in funzione dell'ora del giorno - Gestione dinamica del singolo incrocio con starvation - $\mu$ casuali - corsia Ovest - Sinistra}
  \label{fig:}
\end{figure}
\newpage

Si può notare dalla tabella precedente e dai grafici riportati che per ogni situazione l'algoritmo di gestione dinamica applicato si comporta nettamente meglio di quello relativo ad una gestione statica, comunemente implementato, in una configurazione bilanciata, ovvero con un numero di auto totali che attraversano la giunzione paragonabile per ogni corsia. 

Nello specifico è interessante notare come vengano abbattuti drasticamente i tempi di attesa medi delle auto, e come nelle corsie più affollate anche il massimo numero di auto in coda raggiunto sia nettamente inferiore con la gestione proposta. Questo comporta innumerevoli benefici al traffico cittadino, meno congestionanto e quindi, di conseguenza, anche con meno emissioni di scarichi nocivi. 

Analizzando la corsia \textit{Ovest - Sinistra}, ovvero relativa alle auto che dal braccio ovest vogliono svoltare a sinistra, presa come esempio nella generazione dei grafici, in particolare, è facile comprendere che il massimo tempo di attesa raggiunto con la gestione statica sia di ben $80s$, a fronte dei $25s$ dei due algoritmi dinamici, così come è ben rilevabile che il massimo numero di auto in coda passi da 25 a 11 / 13.

Confrontando poi le gestioni con e senza \textit{starvation} si nota come le due siano essenzialmente equivalenti. Questo sia perché la gestione della \textit{starvation} interviene molto raramente, sia perché comunque, anche gestendo questa particolare variabile, l'algoritmo resta ottimizzato rispetto alla ricerca della corsia complementare a quella in attesa.

Queste considerazioni valgono per qualsiasi corsia, come emerge dalla tabella riportata, ed è da sottolineare che sono stati generati solo i grafici relativi ad una singola strada per praticità, mentre tutti i valori ottenuti sono stati fedelmente trascritti nella suddetta tabella.
\newline

Nella seguente simulazione ci si è concentrati sulla modellazione di una configurazione sbilanciata, con strade molto più affollate di altre, per comprendere anche in questo caso quanto sia pratico utilizzare un algoritmo di gestione dinamica. Nello specifico le corsie relative ai bracci \textit{Nord} e \textit{Sud} sono state volutamente congestionate, fissando il valore del parametro $\mu$ a 25, di base, a cui, si ricorda, deve essere sottratto un numero pari a 6 durante le ore di punta (mattina e pranzo) ed un numero pari a 3 durante l'ora di cena, mentre viene aggiunto un valore pari ad 80 durante la notte, per simulare il normale afflusso dei veicoli rispetto allo specifico periodo della giornata. Una spiegazione più approfondita di tutto ciò è stata già data nel \textit{Capitolo \ref{capitolo1}}. Le strade dei bracci \textit{Est} ed \textit{Ovest}, invece sono state lasciate abbastanza libere ($\mu$ pari a 60).




\begin{table}[H]
\centering
\begin{tabular}{|C{2cm}|C{1.65cm}|C{1.65cm}|C{1.65cm}|C{1.65cm}|C{1.65cm}|C{1.65cm}|}
  \hline
  \multicolumn{7}{|c|}{\large Seconda simulazione: prime quattro corsie} \\\hline\hline
  \multirow{7}{2cm}{\centering \textit{\footnotesize{Corsia Ovest \newline --- \newline Sinistra}}}
  & \multicolumn{2}{C{3.7cm}|}{\textbf{$\mu$}} 
  & \multicolumn{2}{C{3.7cm}|}{\textbf{\footnotesize{Tempo di arrivo medio}}} 
  & \multicolumn{2}{C{3.7cm}|}{\textbf{\footnotesize{Numero di auto totale}}} \\\cline{2-7}
  & \multicolumn{2}{C{3.7cm}|}{60} 
  & \multicolumn{2}{C{3.7cm}|}{$41.59s$} 
  & \multicolumn{2}{C{3.7cm}|}{1238} \\\cline{2-7}
  & \multicolumn{2}{C{3.7cm}|}{\footnotesize{\textbf{Gestione statica}}} 
  & \multicolumn{2}{C{3.7cm}|}{\footnotesize{\textbf{Gestione dinamica senza starvation}}} 
  & \multicolumn{2}{C{3.7cm}|}{\footnotesize{\textbf{Gestione dinamica con starvation}}} \\\cline{2-7}
  & \scriptsize{Max numero auto in coda}
  & \scriptsize{Tempo di attesa medio}
  & \scriptsize{Max numero auto in coda}
  & \scriptsize{Tempo di attesa medio}
  & \scriptsize{Max numero auto in coda}
  & \scriptsize{Tempo di attesa medio}\\\cline{2-7}
  & 7
  & $29.81s$
  & 5
  & $14.93s$
  & 5
  & $14.36s$\\\hline\hline
    \multirow{7}{2cm}{\centering \textit{\footnotesize{Corsia Ovest \newline --- \newline Dritto/Destra}}}
  & \multicolumn{2}{C{3.7cm}|}{\textbf{$\mu$}} 
  & \multicolumn{2}{C{3.7cm}|}{\textbf{\footnotesize{Tempo di arrivo medio}}} 
  & \multicolumn{2}{C{3.7cm}|}{\textbf{\footnotesize{Numero di auto totale}}} \\\cline{2-7}
  & \multicolumn{2}{C{3.7cm}|}{60} 
  & \multicolumn{2}{C{3.7cm}|}{$41.59s$} 
  & \multicolumn{2}{C{3.7cm}|}{1237} \\\cline{2-7}
  & \multicolumn{2}{C{3.7cm}|}{\footnotesize{\textbf{Gestione statica}}} 
  & \multicolumn{2}{C{3.7cm}|}{\footnotesize{\textbf{Gestione dinamica senza starvation}}} 
  & \multicolumn{2}{C{3.7cm}|}{\footnotesize{\textbf{Gestione dinamica con starvation}}} \\\cline{2-7}
  & \scriptsize{Max numero auto in coda}
  & \scriptsize{Tempo di attesa medio}
  & \scriptsize{Max numero auto in coda}
  & \scriptsize{Tempo di attesa medio}
  & \scriptsize{Max numero auto in coda}
  & \scriptsize{Tempo di attesa medio}\\\cline{2-7}
  & 6
  & $29.56s$
  & 5
  & $17.30s$
  & 5
  & $17.29s$\\\hline\hline
  \multirow{7}{2cm}{\centering \textit{\footnotesize{Corsia Nord \newline --- \newline Sinistra}}}
  & \multicolumn{2}{C{3.7cm}|}{\textbf{$\mu$}} 
  & \multicolumn{2}{C{3.7cm}|}{\textbf{\footnotesize{Tempo di arrivo medio}}} 
  & \multicolumn{2}{C{3.7cm}|}{\textbf{\footnotesize{Numero di auto totale}}} \\\cline{2-7}
  & \multicolumn{2}{C{3.7cm}|}{25} 
  & \multicolumn{2}{C{3.7cm}|}{$17.33s$} 
  & \multicolumn{2}{C{3.7cm}|}{2859} \\\cline{2-7}
  & \multicolumn{2}{C{3.7cm}|}{\footnotesize{\textbf{Gestione statica}}} 
  & \multicolumn{2}{C{3.7cm}|}{\footnotesize{\textbf{Gestione dinamica senza starvation}}} 
  & \multicolumn{2}{C{3.7cm}|}{\footnotesize{\textbf{Gestione dinamica con starvation}}} \\\cline{2-7}
  & \scriptsize{Max numero auto in coda}
  & \scriptsize{Tempo di attesa medio}
  & \scriptsize{Max numero auto in coda}
  & \scriptsize{Tempo di attesa medio}
  & \scriptsize{Max numero auto in coda}
  & \scriptsize{Tempo di attesa medio}\\\cline{2-7}
  & 17
  & $57.01s$
  & 7
  & $14.31s$
  & 7
  & $14.05s$\\\hline\hline
  \multirow{7}{2cm}{\centering \textit{\footnotesize{Corsia Nord \newline --- \newline Dritto/Destra}}}
  & \multicolumn{2}{C{3.7cm}|}{\textbf{$\mu$}} 
  & \multicolumn{2}{C{3.7cm}|}{\textbf{\footnotesize{Tempo di arrivo medio}}} 
  & \multicolumn{2}{C{3.7cm}|}{\textbf{\footnotesize{Numero di auto totale}}} \\\cline{2-7}
  & \multicolumn{2}{C{3.7cm}|}{25} 
  & \multicolumn{2}{C{3.7cm}|}{$17.33s$} 
  & \multicolumn{2}{C{3.7cm}|}{2872} \\\cline{2-7}
  & \multicolumn{2}{C{3.7cm}|}{\footnotesize{\textbf{Gestione statica}}} 
  & \multicolumn{2}{C{3.7cm}|}{\footnotesize{\textbf{Gestione dinamica senza starvation}}} 
  & \multicolumn{2}{C{3.7cm}|}{\footnotesize{\textbf{Gestione dinamica con starvation}}} \\\cline{2-7}
  & \scriptsize{Max numero auto in coda}
  & \scriptsize{Tempo di attesa medio}
  & \scriptsize{Max numero auto in coda}
  & \scriptsize{Tempo di attesa medio}
  & \scriptsize{Max numero auto in coda}
  & \scriptsize{Tempo di attesa medio}\\\cline{2-7}
  & 14
  & $51.36s$
  & 7
  & $12.83s$
  & 7
  & $13.18s$\\\hline
\end{tabular}
\caption{Tabella di comparazione fra algoritmi di gestione del singolo incrocio
- prime quattro corsie - configurazione sbilanciata}
\label{table:keytable}
\end{table}
\newpage

\begin{table}[H]
\centering
\begin{tabular}{|C{2cm}|C{1.65cm}|C{1.65cm}|C{1.65cm}|C{1.65cm}|C{1.65cm}|C{1.65cm}|}
  \hline
  \multicolumn{7}{|c|}{\large Seconda simulazione: ultime quattro corsie} \\\hline\hline
  \multirow{7}{2cm}{\centering \textit{\footnotesize{Corsia Est \newline --- \newline Sinistra}}}
  & \multicolumn{2}{C{3.7cm}|}{\textbf{$\mu$}} 
  & \multicolumn{2}{C{3.7cm}|}{\textbf{\footnotesize{Tempo di arrivo medio}}} 
  & \multicolumn{2}{C{3.7cm}|}{\textbf{\footnotesize{Numero di auto totale}}} \\\cline{2-7}
  & \multicolumn{2}{C{3.7cm}|}{60} 
  & \multicolumn{2}{C{3.7cm}|}{$41.59s$} 
  & \multicolumn{2}{C{3.7cm}|}{1226} \\\cline{2-7}
  & \multicolumn{2}{C{3.7cm}|}{\footnotesize{\textbf{Gestione statica}}} 
  & \multicolumn{2}{C{3.7cm}|}{\footnotesize{\textbf{Gestione dinamica senza starvation}}} 
  & \multicolumn{2}{C{3.7cm}|}{\footnotesize{\textbf{Gestione dinamica con starvation}}} \\\cline{2-7}
  & \scriptsize{Max numero auto in coda}
  & \scriptsize{Tempo di attesa medio}
  & \scriptsize{Max numero auto in coda}
  & \scriptsize{Tempo di attesa medio}
  & \scriptsize{Max numero auto in coda}
  & \scriptsize{Tempo di attesa medio}\\\cline{2-7}
  & 5
  & $32.61s$
  & 4
  & $17.11s$
  & 4
  & $17.74s$\\\hline\hline
    \multirow{7}{2cm}{\centering \textit{\footnotesize{Corsia Est \newline --- \newline Dritto/Destra}}}
  & \multicolumn{2}{C{3.7cm}|}{\textbf{$\mu$}} 
  & \multicolumn{2}{C{3.7cm}|}{\textbf{\footnotesize{Tempo di arrivo medio}}} 
  & \multicolumn{2}{C{3.7cm}|}{\textbf{\footnotesize{Numero di auto totale}}} \\\cline{2-7}
  & \multicolumn{2}{C{3.7cm}|}{60} 
  & \multicolumn{2}{C{3.7cm}|}{$41.59s$} 
  & \multicolumn{2}{C{3.7cm}|}{1254} \\\cline{2-7}
  & \multicolumn{2}{C{3.7cm}|}{\footnotesize{\textbf{Gestione statica}}} 
  & \multicolumn{2}{C{3.7cm}|}{\footnotesize{\textbf{Gestione dinamica senza starvation}}} 
  & \multicolumn{2}{C{3.7cm}|}{\footnotesize{\textbf{Gestione dinamica con starvation}}} \\\cline{2-7}
  & \scriptsize{Max numero auto in coda}
  & \scriptsize{Tempo di attesa medio}
  & \scriptsize{Max numero auto in coda}
  & \scriptsize{Tempo di attesa medio}
  & \scriptsize{Max numero auto in coda}
  & \scriptsize{Tempo di attesa medio}\\\cline{2-7}
  & 6
  & $33.42s$
  & 5
  & $21.91s$
  & 5
  & $21.05s$\\\hline\hline
  \multirow{7}{2cm}{\centering \textit{\footnotesize{Corsia Sud \newline --- \newline Sinistra}}}
  & \multicolumn{2}{C{3.7cm}|}{\textbf{$\mu$}} 
  & \multicolumn{2}{C{3.7cm}|}{\textbf{\footnotesize{Tempo di arrivo medio}}} 
  & \multicolumn{2}{C{3.7cm}|}{\textbf{\footnotesize{Numero di auto totale}}} \\\cline{2-7}
  & \multicolumn{2}{C{3.7cm}|}{25} 
  & \multicolumn{2}{C{3.7cm}|}{$17.33s$} 
  & \multicolumn{2}{C{3.7cm}|}{2875} \\\cline{2-7}
  & \multicolumn{2}{C{3.7cm}|}{\footnotesize{\textbf{Gestione statica}}} 
  & \multicolumn{2}{C{3.7cm}|}{\footnotesize{\textbf{Gestione dinamica senza starvation}}} 
  & \multicolumn{2}{C{3.7cm}|}{\footnotesize{\textbf{Gestione dinamica con starvation}}} \\\cline{2-7}
  & \scriptsize{Max numero auto in coda}
  & \scriptsize{Tempo di attesa medio}
  & \scriptsize{Max numero auto in coda}
  & \scriptsize{Tempo di attesa medio}
  & \scriptsize{Max numero auto in coda}
  & \scriptsize{Tempo di attesa medio}\\\cline{2-7}
  & 15
  & $53.73s$
  & 6
  & $17.69s$
  & 6
  & $17.86s$\\\hline\hline
  \multirow{7}{2cm}{\centering \textit{\footnotesize{Corsia Sud \newline --- \newline Dritto/Destra}}}
  & \multicolumn{2}{C{3.7cm}|}{\textbf{$\mu$}} 
  & \multicolumn{2}{C{3.7cm}|}{\textbf{\footnotesize{Tempo di arrivo medio}}} 
  & \multicolumn{2}{C{3.7cm}|}{\textbf{\footnotesize{Numero di auto totale}}} \\\cline{2-7}
  & \multicolumn{2}{C{3.7cm}|}{25} 
  & \multicolumn{2}{C{3.7cm}|}{$17.33s$} 
  & \multicolumn{2}{C{3.7cm}|}{2885} \\\cline{2-7}
  & \multicolumn{2}{C{3.7cm}|}{\footnotesize{\textbf{Gestione statica}}} 
  & \multicolumn{2}{C{3.7cm}|}{\footnotesize{\textbf{Gestione dinamica senza starvation}}} 
  & \multicolumn{2}{C{3.7cm}|}{\footnotesize{\textbf{Gestione dinamica con starvation}}} \\\cline{2-7}
  & \scriptsize{Max numero auto in coda}
  & \scriptsize{Tempo di attesa medio}
  & \scriptsize{Max numero auto in coda}
  & \scriptsize{Tempo di attesa medio}
  & \scriptsize{Max numero auto in coda}
  & \scriptsize{Tempo di attesa medio}\\\cline{2-7}
  & 20
  & $55.14s$
  & 6
  & $16.86s$
  & 8
  & $17.77s$\\\hline
\end{tabular}
\caption{Tabella di comparazione fra algoritmi di gestione del singolo incrocio
- ultime quattro corsie - configurazione sbilanciata}
\label{table:keytable}
\end{table}

Di seguito sono riportati alcuni grafici relativi alla corsia \textit{Ovest - Sinistra}, poco affollata, ed a quella \textit{Nord - Sinistra}, più congestionata.
\newpage
\begin{figure}[H]
\centering
  \includegraphics[width=0.9\textwidth]{figuraWestSinistraPartiti.pdf}
  \caption{Auto che hanno attraversato la corsia Ovest - Sinistra (uguale per i tre modelli di gestione) - configurazione sbilanciata}
  \label{fig:partitiMuSbil}
\end{figure}
\begin{figure}[H]
\centering
  \includegraphics[width=0.9\textwidth]{figuraWestSinistraCodaGestioneStatica.pdf}
  \caption{Numero di auto in coda in funzione dell'ora del giorno - Gestione statica del singolo incrocio - Configurazione sbilanciata - corsia Ovest - Sinistra}
  \label{fig:}
\end{figure}

\newpage

\begin{figure}[H]
\centering
  \includegraphics[width=0.9\textwidth]{figuraWestSinistraAttesaGestioneStatica.pdf}
  \caption{Tempi di attesa in funzione dell'ora del giorno - Gestione statica del singolo incrocio - Configurazione sbilanciata - corsia Ovest - Sinistra}
  \label{fig:}
\end{figure}
\begin{figure}[H]
\centering
  \includegraphics[width=0.9\textwidth]{figuraWestSinistraCodaGestioneDinamica.pdf}
  \caption{Numero di auto in coda in funzione dell'ora del giorno - Gestione dinamica del singolo incrocio senza starvation - Configurazione sbilanciata - corsia Ovest - Sinistra}
  \label{fig:}
\end{figure}

\newpage

\begin{figure}[H]
\centering
  \includegraphics[width=0.9\textwidth]{figuraWestSinistraAttesaGestioneDinamica.pdf}
  \caption{Tempi di attesa in funzione dell'ora del giorno - Gestione dinamica del singolo incrocio senza starvation - Configurazione sbilanciata - corsia Ovest - Sinistra}
  \label{fig:}
\end{figure}
\begin{figure}[H]
\centering
  \includegraphics[width=0.9\textwidth]{figuraWestSinistraCodaGestioneDinamicaStarvation.pdf}
  \caption{Numero di auto in coda in funzione dell'ora del giorno - Gestione dinamica del singolo incrocio con starvation - Configurazione sbilanciata - corsia Ovest - Sinistra}
  \label{fig:}
\end{figure}
\newpage
\begin{figure}[H]
\centering
  \includegraphics[width=0.9\textwidth]{figuraWestSinistraAttesaGestioneDinamicaStarvation.pdf}
  \caption{Tempi di attesa in funzione dell'ora del giorno - Gestione dinamica del singolo incrocio con starvation - Configurazione sbilanciata - corsia Ovest - Sinistra}
  \label{fig:}
\end{figure}

\newpage
\begin{figure}[H]
\centering
  \includegraphics[width=0.9\textwidth]{figuraNordSinistraPartiti.pdf}
  \caption{Auto che hanno attraversato la corsia Nord - Sinistra (uguale per i tre modelli di gestione) - configurazione sbilanciata}
  \label{fig:partitiMuSbil}
\end{figure}
\begin{figure}[H]
\centering
  \includegraphics[width=0.9\textwidth]{figuraNordSinistraCodaGestioneStatica.pdf}
  \caption{Numero di auto in coda in funzione dell'ora del giorno - Gestione statica del singolo incrocio - Configurazione sbilanciata - corsia Nord - Sinistra}
  \label{fig:}
\end{figure}

\newpage

\begin{figure}[H]
\centering
  \includegraphics[width=0.9\textwidth]{figuraNordSinistraAttesaGestioneStatica.pdf}
  \caption{Tempi di attesa in funzione dell'ora del giorno - Gestione statica del singolo incrocio - Configurazione sbilanciata - corsia Nord - Sinistra}
  \label{fig:}
\end{figure}
\begin{figure}[H]
\centering
  \includegraphics[width=0.9\textwidth]{figuraNordSinistraCodaGestioneDinamica.pdf}
  \caption{Numero di auto in coda in funzione dell'ora del giorno - Gestione dinamica del singolo incrocio senza starvation - Configurazione sbilanciata - corsia Nord - Sinistra}
  \label{fig:}
\end{figure}

\newpage

\begin{figure}[H]
\centering
  \includegraphics[width=0.9\textwidth]{figuraNordSinistraAttesaGestioneDinamica.pdf}
  \caption{Tempi di attesa in funzione dell'ora del giorno - Gestione dinamica del singolo incrocio senza starvation - Configurazione sbilanciata - corsia Nord - Sinistra}
  \label{fig:}
\end{figure}
\begin{figure}[H]
\centering
  \includegraphics[width=0.9\textwidth]{figuraNordSinistraCodaGestioneDinamicaStarvation.pdf}
  \caption{Numero di auto in coda in funzione dell'ora del giorno - Gestione dinamica del singolo incrocio con starvation - Configurazione sbilanciata - corsia Nord - Sinistra}
  \label{fig:}
\end{figure}
\newpage
\begin{figure}[H]
\centering
  \includegraphics[width=0.9\textwidth]{figuraNordSinistraAttesaGestioneDinamicaStarvation.pdf}
  \caption{Tempi di attesa in funzione dell'ora del giorno - Gestione dinamica del singolo incrocio con starvation - Configurazione sbilanciata - corsia Nord - Sinistra}
  \label{fig:}
\end{figure}

Anche in questo caso si notano gli enormi benefici apportati da una gestione intelligente dell'incrocio, a seguito di un'implementazione di un algoritmo come quello proposto. In particolare appare chiaro come non solo vengano abbattuti i tempi di attesa, sia per le corsie affollate che per quelle con meno automobili, ma anche come questi vengano resi pressoché omogenei per tutte le corsie, e come l'ingorgo sia gestito egregiamente.

Per quanto concerne il numero massimo di auto in coda, si ha un sostanziale decremento per le strade più congestionate, mentre questo rimane stabile per quelle con meno entità, a fronte però di attese ridotte. Anche in questo caso la gestione della \textit{starvation} non penalizza lo smaltimento delle automobili.
\newline

In conclusione, dopo numerosi test svolti, si può affermare con certezza che l'algoritmo presentato sia funzionante e gestisca decisamente meglio gli incroci, in qualsiasi condizione, rispetto a quello tradizionale. Chiaramente bisogna considerare il fatto che si è esclusivamente analizzato il comportamento del codice presentato quando applicato ad una singola giunzione, quindi sorge spontanea la domanda: "cosa succederebbe se questo venisse applicato ad un cluster di incroci interconnessi fra loro? I benefici sarebbero altrettanto notevoli?". Per la risposta a questo legittimo interrogativo, nei prossimi capitoli verrà presentato un modello a 9 incroci interconnessi e comunicanti, basato su quanto già esposto, e sarà dimostrato come anche in una simulazione di questo genere sia indubbiamente conveniente gestire tali intersezioni dinamicamente.



























\chapter{Modellazione e gestione statica di un cluster di nove incroci interconnessi}
In questo terzo capitolo viene presentato il modello utilizzato per la simulazione di un cluster di nove incroci interconnessi. Tutti questi incroci sono a raso a quattro bracci, esattamente come la singola giunzione vista in precedenza, ma la particolarità è che, essendo collegati, i veicoli che fluiscono da un incrocio entreranno a far parte di una delle code di uno degli incroci successivi. In questo modo si vuole analizzare se una gestione dinamica, come quella presentata, effettivamente riesca ad apportare dei benefici in un contesto più ampio di quello della singola intersezione, o se, al contrario, efficientare il flusso di veicoli di una giunzione congestiona pesantemente le altre.

La base di partenza è, ovviamente, il modello del singolo incrocio realizzato ed ampiamente commentato in precedenza, che è stato trasformato in un sottosistema. Materialmente lo si è inserito in un blocco \textit{Subsystem}\cite{subsystem}, anche questo già visto in precedenza, ed al posto degli \textit{Entity Generator}, sono state collegate alle code delle \textit{Input Port}.

Anche i terminator sono stati eliminati dal blocco, sostituiti invece da delle \textbf{Output Port \cite{outputport}}. Tutto questo è stato fatto in modo tale da poter interconnettere gli incroci: queste porte sono infatti poi accessibili dall'esterno del \textit{Subsystem}, in modo tale da far passare le entità da un sottosistema (e dunque da un incrocio) ad un altro.

\begin{figure}[H]
\centering
  \includegraphics[width=0.3\textwidth]{OutputPort.pdf}
  \caption{Output Port}
  \label{fig:}
\end{figure}

Resta fondamentale sottolineare la presenza di un \textbf{Entity Output Switch \cite{switch}} prima delle \textit{Output Port} di quelle corsie che permettono alle automobili presenti all'interno delle stesse sia di andare dritto che di svoltare a destra. Infatti, se è vero che in precedenza non era importante la decisione presa, in quanto riguardava solo l'uscita dal singolo incrocio, in questo caso invece è fondamentale capire quando un'auto "decide" di proseguire dritto o, al contrario, di girare, perché quella stessa vettura poi, a seconda della sua decisione, proseguirà verso un'altra intersezione specifica. 

Per implementare questo meccanismo decisionale ci si è avvalsi, appunto, di un \textit{Entity Output Switch}. Quello che fa questo componente è molto semplice: per ogni entità che lo attraversa, decide secondo un criterio stabilito (una distribuzione o una funzione) su quale porta di output inviarla. Il numero di canali di output viene precedentemente specificato, così come il meccanismo di scelta. 

\begin{figure}[H]
  \includegraphics[width=1\textwidth]{switchEproprieta.pdf}
  \caption{Entity Output Switch e relativa configurazione}
  \label{fig:}
\end{figure}

In questo caso, come si può notare dalla figura precedente, le output ports sono ovviamente due, una relativa all'uscita che collega tale corsia con l'incrocio immediatamente di fronte ad essa (l'automobile ha deciso di proseguire diritto), un'altra inerente al collegamento con la giunzione a destra rispetto alla corsia designata (l'auto ha deciso di svoltare).

La scelta viene effettuata casualmente in maniera equiprobabile e se ne occupa l'Entity Switch stesso. Come spesso accade in informatica, tuttavia, la decisione viene associata alla generazione di un numero casuale, tramite il quale il componente si regola in autonomia. La generazione di questo numero ha bisogno di un Seed, ovvero un altro numero di partenza, questa volta però predeterminato, ovvero dato in input come parametro all'Entity Switch.

In sostanza, in base a questo numero il componente prenderà una decisione per ogni veicolo che lo attraversa. Questo però espone il modello ad un problema: se il Seed fosse sempre lo stesso anche i numeri generati casualmente in realtà sarebbero identici per ogni simulazione, e quindi se è vero che all'interno della stessa simulazione la scelta sarebbe casuale, è anche vero che fra due simulazioni consecutive in realtà il meccanismo sarebbe deterministico, in quanto si potrebbe già sapere che strada prenderà ciascun veicolo, sulla base del comportamento precedente.

Per spiegarsi meglio, ecco un esempio molto banale: nella ipotetica \textbf{Simulazione A} il \textit{Veicolo 1} della corsia che si sta prendendo in considerazione sceglie di andare dritto. Questo non fornisce alcuna informazione sulla decisione che  prenderà il \textit{Veicolo 2}, e quindi sembra che tutto sia casuale. Si ipotizzi che il \textit{Veicolo 2} decida di svoltare a destra.

In una futura \textbf{Simulazione B}, però, se il Seed inerente a quello specifico Entity Output Switch non è cambiato, si avrà la matematica certezza che il \textit{Veicolo 1} sceglierà di proseguire dritto, e che il \textit{Veicolo 2}, invece, svolterà. Questo rappresenta ovviamente un problema, perché si vuole rendere il modello assolutamente casuale, in modo tale da poter effettuare simulazioni diverse fra loro per validare l'ipotesi per cui una gestione dinamica è valida anche in questo tipo di ambiente.
\newline

Per risolvere il problema, i Seed di ogni Entity Output Switch sono stati generati casualmente all'avvio della simulazione, utilizzando questa volta la funzione \textit{Randi} messa a disposizione da Matlab, inserita nella sezione InitFcn del modello (si veda la \textit{figura \ref{fig:initfcn}}).
Il codice utilizzato è il seguente.
\newline
\begin{lstlisting}[language=Matlab,label=seedgen,caption=Generazione Casuale dei Seed]
    coder.extrinsic('randi');
	seeds = 1;
	seeds = randi(10000, [1 44]); 
\end{lstlisting}

In definitiva, il modello della singola corsia (dritto / destra) è quello in figura. Come si può notare non è stato cambiato nulla che non sia stato citato in questa sua breve descrizione. 

\begin{figure}[H]
  \includegraphics[width=1\textwidth]{catenaIncrocioModificato.pdf}
  \caption{Singola corsia (dritto / destra) di una giunzione del cluster}
  \label{fig:}
\end{figure}

\newpage
\section{La modellazione del cluster}
Compresi i cambiamenti che sono stati apportati al modello del singolo incrocio, è bene descrivere come si è creato il cluster delle nove giunzioni interconnesse.

In primo luogo, ogni sottosistema dispone, ovviamente, di otto porte di input, corrispondenti alle otto corsie dell'incrocio, e di dodici porte di output. In \textit{figura \ref{fig:switchstreet}} si può notare che sono stati utilizzati degli \textbf{Entity input switch \cite{inpswitch}}, elementi leggermente diversi dagli \textit{Entity Output Switch}, in quanto presentano un numero di ingressi multiplo ed un'unica uscita. Essi sono stati poi collegati a degli \textit{Output Switch}.

\begin{figure}[H]
\centering
  \includegraphics[width=0.5\textwidth]{switchStreet.pdf}
  \caption{Collegamento fra Entity Input Switch ed Entity Output Switch}
  \label{fig:switchstreet}
\end{figure}

Questo per un motivo molto semplice: considerando due incroci adiacenti, A e B, ci sono tre corsie di uscita da A che vanno verso B, e solo due ingressi per B (quelli del braccio collegato ad A). Un'analisi della \textit{figura \ref{fig:ingrandimentoincroci}} è più esplicativa di mille parole. 

Prendendo per esempio proprio la figura seguente, si chiami A l'incrocio più a sinistra e B quello più a destra. Le corsie di uscita da A che vanno verso B sono:
\begin{enumerate}
  \item Braccio Nord - Corsia per svoltare a sinistra
  \item Braccio Sud - Corsia per svoltare a destra
  \item Braccio Ovest - Corsia per andare dritto
\end{enumerate}
Mentre per quanto concerne le corsie di B che sono collegate ad A, queste sono solo quelle del braccio Ovest, quella per andare dritto / svoltare a destra e quella per svoltare a sinistra.
\begin{figure}[H]
  \includegraphics[width=1\textwidth]{collegamentoIncroci.pdf}
  \caption{Ingrandimento del raccordo fra due incroci adiacenti}
  \label{fig:ingrandimentoincroci}
\end{figure}

Ecco dunque spiegata l'utilità degli Switch, che servono a redistribuire i veicoli provenienti da A nelle corsie di B adiacenti, anche questa volta utilizzando un Seed generato casualmente per ogni esecuzione. Ogni \textit{Input Switch}, nello specifico, si occupa di accorpare il flusso di autovetture provenienti dalle tre corsie di output di un incrocio, e di inviare questo flusso ad un \textit{Output Switch}, che smista le automobili nelle due strade di input dell'incrocio successivo.
\newline

Ovviamente queste automobili devono essere in qualche modo generate, e devono in qualche modo fuoriuscire dal cluster. Per questo gli incroci più esterni, in alcune delle loro corsie, sono stati collegati a degli Entity Generator, programmati esattamente come visto per il modello a singolo incrocio, e a degli Entity Terminator. È palese che la struttura creata sia frattalica ed ampiamente espandibile: nulla impedisce di creare un cluster di un numero di incroci nettamente superiore a nove, ma lo scopo di questa tesi non è quello di mappare le intersezioni stradali di un'intera città in un modello di SimEvents.

È anche importante notare che in questo caso la capacità delle code che rappresentano le corsie non è stata lasciata ad infinito, essendo in realtà rappresentativa della distanza fra due giunzioni consecutive. Pertanto si è deciso di operare in questo modo: le code direttamente collegate a degli \textit{Entity Generator} sono state volontariamente programmate per avere lunghezza infinita, in quanto altrimenti si sarebbero potute riempire in situazioni di traffico intenso non dando la possibilità ai generator di funzionare a dovere. Per quelle che invece rappresentano una corsia che collega un incrocio ad un altro, la loro capienza è stata impostata ad un numero massimo, il cui valore specifico sarà poi reso noto nel paragrafo relativo alle simulazioni effettuate.

In definitiva, il modello a nove incroci è presentato nella figura seguente.
\newpage

\begin{figure}[H]
  \includegraphics[width=1\textwidth]{capitolo3.pdf}
  \caption{Cluster di nove incroci interconnessi}
  \label{fig:}
\end{figure}
\newpage

\section{Risultati simulazione con gestione statica}

In questa simulazione si vuole mettere in evidenza che, in un modello di questo genere, gli ingressi del cluster (facilmente assimilabili alle grandi strade di ingresso ad una città / quartiere), si congestionano molto facilmente per via del limite massimo di macchine che, come spiegato, è stato assegnato ad ogni strada, eccezion fatta per quelle connesse ai Generator. 

Nello specifico questo limite è stato posto a 10. Per quanto concerne i tassi di generazione, il parametro $\mu$ varia fra 25 e 35 in maniera uniforme, analogamente a quanto visto in precedenza. Nessun altro parametro di configurazione è stato modificato.

Nella tabella, i nove incroci del cluster sono stati visti come una matrice (3x3), quindi la giunzione [1-1] è quella in alto a sinistra, la [1-2] quella immediatamente successiva sulla stessa riga e così via. Per esempio, le prime informazioni riportate alla pagina seguente appartengono alla corsia del braccio ovest, dell'incrocio in alto a sinistra, riservata alle macchine che vogliono proseguire dritto (verso l'incrocio [1-2]) o svoltare a destra (verso l'incrocio [2-1]).

Bisogna anche notare la colonna \textit{tipologia}: si è voluto mettere in risalto infatti la tipologia di strada che si sta considerando, se essa è collegata ad un Generator, ad un Terminator, oppure se è una delle corsie di collegamento tra un incrocio ed un altro. Ovviamente solo le strade collegate ad un Generator, come spiegato, possono superare 10 come numero massimo di macchine in coda.
\begin{table}[H]
\centering
\begin{tabular}{|C{2cm}|C{2cm}|C{3cm}|C{3cm}|C{3cm}|}
\hline
\textbf{Corsia} &
\textbf{Tipologia} &
\textbf{Max numero auto in coda} &
\textbf{Tempo di attesa medio} &
\textbf{Numero di auto totale} \\\hline
\scriptsize{[1 - 1] - Ovest - Dritto-Destra} &
Generator &
12 &
42.81 &
2550 \\\hline
\scriptsize{[1 - 1] - Ovest - Sinistra} &
Generator &
15 &
49.27 &
2647 \\\hline
\scriptsize{[1 - 1] - Nord - Dritto-Destra} &
Generator &
11 &
38.01 &
2301 \\\hline
\scriptsize{[1 - 1] - Nord - Sinistra} &
Generator &
12 &
37.63 &
2402 \\\hline
\scriptsize{[1 - 1] - Est - Dritto-Destra} &
Terminator &
9 &
21.96 &
2346 \\\hline
\scriptsize{[1 - 1] - Est - Sinistra} &
Intermedio &
10 &
21.95 &
2402 \\\hline
\scriptsize{[1 - 1] - Sud - Dritto-Destra} &
Terminator &
9 &
23.79 &
2427 \\\hline
\scriptsize{[1 - 1] - Sud - Sinistra} &
Terminator &
10 &
27.48 &
2416 \\\hline
\scriptsize{[1 - 2] - Ovest - Dritto-Destra} &
Intermedio &
10 &
52.48 &
2510 \\\hline
\scriptsize{[1 - 2] - Ovest - Sinistra} &
Terminator &
9 &
45.73 &
2264 \\\hline
\scriptsize{[1 - 2] - Nord - Dritto-Destra} &
Generator &
11 &
38.84 &
2342 \\\hline
\scriptsize{[1 - 2] - Nord - Sinistra} &
Generator &
9 &
35.20 &
2219 \\\hline
\scriptsize{[1 - 2] - Est - Dritto-Destra} &
Intermedio &
10 &
23.95 &
2306 \\\hline
\scriptsize{[1 - 2] - Est - Sinistra} &
Intermedio &
7 &
22.83 &
2329 \\\hline
\scriptsize{[1 - 2] - Sud - Dritto-Destra} &
Terminator &
10 &
52.39 &
2488 \\\hline
\scriptsize{[1 - 2] - Sud - Sinistra} &
Intermedio &
10 &
46.14 &
2407 \\\hline
\end{tabular}
\caption{Cluster di nove incroci interconnessi - simulazione con gestione statica - pt. 1}
\label{table:keytable}
\end{table}
\newpage
\begin{table}[H]
\centering
\begin{tabular}{|C{2cm}|C{2cm}|C{3cm}|C{3cm}|C{3cm}|}
\hline
\textbf{Corsia} &
\textbf{Tipologia} &
\textbf{Max numero auto in coda} &
\textbf{Tempo di attesa medio} &
\textbf{Numero di auto totale} \\\hline
\scriptsize{[1 - 3] - Ovest - Dritto-Destra} &
Terminator &
10 &
46.32 &
2330 \\\hline
\scriptsize{[1 - 3] - Ovest - Sinistra} &
Terminator &
10 &
47.46 &
2311 \\\hline
\scriptsize{[1 - 3] - Nord - Dritto-Destra} &
Generator &
8 &
36.24 &
2143 \\\hline
\scriptsize{[1 - 3] - Nord - Sinistra} &
Generator &
11 &
41.70 &
2333 \\\hline
\scriptsize{[1 - 3] - Est - Dritto-Destra} &
Generator &
21 &
44.85 &
2660 \\\hline
\scriptsize{[1 - 3] - Est - Sinistra} &
Generator &
9 &
38.46 &
2123 \\\hline
\scriptsize{[1 - 3] - Sud - Dritto-Destra} &
Terminator &
10 &
25.90 &
2416 \\\hline
\scriptsize{[1 - 3] - Sud - Sinistra} &
Intermedio &
10 &
30.20 &
2450 \\\hline
\scriptsize{[2 - 1] - Ovest - Dritto-Destra} &
Generator &
11 &
40.34 &
2358 \\\hline
\scriptsize{[2 - 1] - Ovest - Sinistra} &
Generator &
7 &
36.91 &
2052 \\\hline
\scriptsize{[2 - 1] - Nord - Dritto-Destra} &
Intermedio &
10 &
48.55 &
2410 \\\hline
\scriptsize{[2 - 1] - Nord - Sinistra} &
Intermedio &
9 &
45.89 &
2470 \\\hline
\scriptsize{[2 - 1] - Est - Dritto-Destra} &
Terminator &
9 &
48.96 &
2401 \\\hline
\scriptsize{[2 - 1] - Est - Sinistra} &
Intermedio &
10 &
48.44 &
2401 \\\hline
\scriptsize{[2 - 1] - Sud - Dritto-Destra} &
Intermedio &
10 &
49.39 &
2640 \\\hline
\scriptsize{[2 - 1] - Sud - Sinistra} &
Terminator &
10 &
48.50 &
2485 \\\hline
\end{tabular}
\caption{Cluster di nove incroci interconnessi - simulazione con gestione statica - pt. 2}
\label{table:keytable}
\end{table}
\newpage
\begin{table}[H]
\centering
\begin{tabular}{|C{2cm}|C{2cm}|C{3cm}|C{3cm}|C{3cm}|}
\hline
\textbf{Corsia} &
\textbf{Tipologia} &
\textbf{Max numero auto in coda} &
\textbf{Tempo di attesa medio} &
\textbf{Numero di auto totale} \\\hline
\scriptsize{[2 - 2] - Ovest - Dritto-Destra} &
Intermedio &
10 &
24.46 &
2347 \\\hline
\scriptsize{[2 - 2] - Ovest - Sinistra} &
Intermedio &
8 &
23.63 &
2288 \\\hline
\scriptsize{[2 - 2] - Nord - Dritto-Destra} &
Intermedio &
10 &
22.66 &
2222 \\\hline
\scriptsize{[2 - 2] - Nord - Sinistra} &
Intermedio &
9 &
24.08 &
2329 \\\hline
\scriptsize{[2 - 2] - Est - Dritto-Destra} &
Intermedio &
10 &
50.58 &
2510 \\\hline
\scriptsize{[2 - 2] - Est - Sinistra} &
Intermedio &
10 &
54.43 &
2550 \\\hline
\scriptsize{[2 - 2] - Sud - Dritto-Destra} &
Intermedio &
9 &
43.98 &
2364 \\\hline
\scriptsize{[2 - 2] - Sud - Sinistra} &
Intermedio &
10 &
45.38 &
2477 \\\hline
\scriptsize{[2 - 3] - Ovest - Dritto-Destra} &
Terminator &
8 &
23.55 &
2273 \\\hline
\scriptsize{[2 - 3] - Ovest - Sinistra} &
Intermedio &
9 &
23.86 &
2295 \\\hline
\scriptsize{[2 - 3] - Nord - Dritto-Destra} &
Intermedio &
10 &
49.60 &
2510 \\\hline
\scriptsize{[2 - 3] - Nord - Sinistra} &
Terminator &
10 &
52.28 &
2473 \\\hline
\scriptsize{[2 - 3] - Est - Dritto-Destra} &
Generator &
12 &
34.79 &
2076 \\\hline
\scriptsize{[2 - 3] - Est - Sinistra} &
Generator &
16 &
48.05 &
2444 \\\hline
\scriptsize{[2 - 3] - Sud - Dritto-Destra} &
Intermedio &
10 &
60.28 &
2594 \\\hline
\scriptsize{[2 - 3] - Sud - Sinistra} &
Intermedio &
10 &
56.55 &
2659 \\\hline
\end{tabular}
\caption{Cluster di nove incroci interconnessi - simulazione con gestione statica - pt. 3}
\label{table:keytable}
\end{table}
\newpage
\begin{table}[H]
\centering
\begin{tabular}{|C{2cm}|C{2cm}|C{3cm}|C{3cm}|C{3cm}|}
\hline
\textbf{Corsia} &
\textbf{Tipologia} &
\textbf{Max numero auto in coda} &
\textbf{Tempo di attesa medio} &
\textbf{Numero di auto totale} \\\hline
\scriptsize{[3 - 1] - Ovest - Dritto-Destra} &
Generator &
14 &
43.94 &
2403 \\\hline
\scriptsize{[3 - 1] - Ovest - Sinistra} &
Generator &
39 &
76.84 &
2719 \\\hline
\scriptsize{[3 - 1] - Nord - Dritto-Destra} &
Terminator &
10 &
27.69 &
2353 \\\hline
\scriptsize{[3 - 1] - Nord - Sinistra} &
Intermedio &
10 &
22.04 &
2305 \\\hline
\scriptsize{[3 - 1] - Est - Dritto-Destra} &
Terminator &
10 &
56.07 &
2728 \\\hline
\scriptsize{[3 - 1] - Est - Sinistra} &
Terminator &
10 &
52.92 &
2678 \\\hline
\scriptsize{[3 - 1] - Sud - Dritto-Destra} &
Generator &
26 &
85.20 &
2791 \\\hline
\scriptsize{[3 - 1] - Sud - Sinistra} &
Generator &
25 &
57.81 &
2683 \\\hline
\scriptsize{[3 - 2] - Ovest - Dritto-Destra} &
Intermedio &
10 &
31.00 &
2500 \\\hline
\scriptsize{[3 - 2] - Ovest - Sinistra} &
Intermedio &
10 &
40.48 &
2498 \\\hline
\scriptsize{[3 - 2] - Nord - Dritto-Destra} &
Terminator &
10 &
26.28 &
2437 \\\hline
\scriptsize{[3 - 2] - Nord - Sinistra} &
Intermedio &
10 &
24.65 &
2380 \\\hline
\scriptsize{[3 - 2] - Est - Dritto-Destra} &
Intermedio &
10 &
51.64 &
2554 \\\hline
\scriptsize{[3 - 2] - Est - Sinistra} &
Terminator &
10 &
53.59 &
2548 \\\hline
\scriptsize{[3 - 2] - Sud - Dritto-Destra} &
Generator &
31 &
80.12 &
2943 \\\hline
\scriptsize{[3 - 2] - Sud - Sinistra} &
Generator &
16 &
41.90 &
2094 \\\hline
\end{tabular}
\caption{Cluster di nove incroci interconnessi - simulazione con gestione statica - pt. 4}
\label{table:keytable}
\end{table}
\newpage
\begin{table}[H]
\centering
\begin{tabular}{|C{2cm}|C{2cm}|C{3cm}|C{3cm}|C{3cm}|}
\hline
\textbf{Corsia} &
\textbf{Tipologia} &
\textbf{Max numero auto in coda} &
\textbf{Tempo di attesa medio} &
\textbf{Numero di auto totale} \\\hline
\scriptsize{[3 - 3] - Ovest - Dritto-Destra} &
Terminator &
10 &
40.42 &
2383 \\\hline
\scriptsize{[3 - 3] - Ovest - Sinistra} &
Intermedio &
8 &
21.69 &
2318 \\\hline
\scriptsize{[3 - 3] - Nord - Dritto-Destra} &
Terminator &
10 &
25.92 &
2216 \\\hline
\scriptsize{[3 - 3] - Nord - Sinistra} &
Terminator &
8 &
23.73 &
2232 \\\hline
\scriptsize{[3 - 3] - Est - Dritto-Destra} &
Generator &
47 &
104.77 &
2936 \\\hline
\scriptsize{[3 - 3] - Est - Sinistra} &
Generator &
17 &
64.44 &
2847 \\\hline
\scriptsize{[3 - 3] - Sud - Dritto-Destra} &
Generator &
17 &
57.00 &
2559 \\\hline
\scriptsize{[3 - 3] - Sud - Sinistra} &
Generator &
20 &
66.16 &
2837 \\\hline
\end{tabular}
\caption{Cluster di nove incroci interconnessi - simulazione con gestione statica - pt. 5}
\label{table:keytable}
\end{table}

È significativo, in questa tabella, notare lo squilibrio, già preventivato nell'introduzione di questo paragrafo, fra le corsie di "accesso" al cluster, quelle con i Generator, e le altre. Questo squilibrio si riflette in tempi di attesa elevatissimi e code molto lunghe in tali strade. 

Come sarà possibile notare in seguito, in una simulazione esattamente analoga a quella qui riportata, una gestione dinamica del medesimo flusso di macchine, con l'algoritmo già presentato, risolve il problema e riesce a redistribuire i tempi di attesa rendendoli uniformi in tutto il modello, e mediamente molto più bassi.

Sono qui riportati i grafici inerenti alla corsia più congestionata di tutto il modello, per rendere ancora più visibile come il numero di macchine in coda ed il tempo di attesa varino a seconda delle ore del giorno, pur mantenendosi sempre abbastanza elevati.
\newpage

\begin{figure}[H]
\centering
  \includegraphics[width=0.9\textwidth]{autocodacap3.pdf}
  \caption{Variazione nel tempo del numero di auto in coda, corsia più congestionata della simulazione}
  \label{fig:}
\end{figure}
\begin{figure}[H]
\centering
  \includegraphics[width=0.9\textwidth]{tempoattesacap3.pdf}
  \caption{Tempi di attesa in funzione della fascia oraria, corsia più congestionata della simulazione}
  \label{fig:}
\end{figure}
\newpage

\begin{figure}[H]
\centering
  \includegraphics[width=0.9\textwidth]{autototcap3.pdf}
  \caption{Numero complessivo di auto, corsia più congestionata della simulazione}
  \label{fig:}
\end{figure}














\chapter{Gestione del cluster con un algoritmo di controllo dinamico}
In questo capitolo si analizza come la gestione del cluster di nove incroci interconnessi presentato in precedenza, con un algoritmo di controllo dinamico, riesca a ridurre le congestioni, creando un flusso di veicoli più omogeneo, con tempi di attesa minori e generalmente uniformi, e con un numero di macchine in coda, per ogni corsia presente (che sia essa intermedia o collegata ad un Generator) tenuto sotto controllo.

L'algoritmo utilizzato è esattamente quello presentato nel \textit{Capitolo \ref{capitolo2}} (Codice \ref{algoritmodin}). Questo è stato applicato singolarmente ad ogni incrocio, che lavora in maniera autonoma. Dunque ogni giunzione non ha alcuna coscienza di essere inserita in una simulazione più ampia, e ciò migliora l'espandibilità del modello, essendo appunto ogni incrocio indipendente da tutti gli altri.

Riassumendo, non si è voluto che le giunzioni scambiassero dati fra loro per verificare se queste riuscissero autonomamente a sincronizzarsi e regolarsi al meglio, e si è osservato che in effetti la tendenza è proprio questa: anche senza sapere nulla del contesto circostante, ogni incrocio, applicando in autonomia l'algoritmo già più volte spiegato, riesce ad apportare un contributo all'intero cluster in termini di miglioramento della gestione del flusso automobilistico.
\newpage

\section{Il modello di confronto}
Anche in questo caso, per effettuare un confronto fra una gestione "canonica" ed una gestione ottimizzata si è scelto di creare un modello apposito, in cui ogni Generator viene collegato ad un Entity Replicator, che crea due repliche dell'entità in ingresso (l'automobile) inviandone una al cluster gestito staticamente e l'altra a quello i cui incroci implementano l'algoritmo. 

In questo modo i due cluster hanno esattamente gli stessi input, pertanto si può effettuare un confronto puntuale e preciso su chi stia gestendo meglio il flusso e quali parametri stia ottimizzando.
\newline

Nella realizzazione del modello non ci si è discostati da quanto detto nel \textit{Capitolo \ref{capitolo2}} (\textit{Paragrafo 2.4}), semplicemente si è esteso il concetto presentato in precedenza all'intero cluster ed a tutti gli Entity Generator presenti. Per questo motivo non si entrerà nello specifico nella spiegazione del suddetto modello, il cui schema è riportato alla pagina seguente.
\newpage

\section{I risultati ottenuti}
Nel precedente capitolo è stata analizzata una simulazione considerando esclusivamente i risultati ottenuti con una gestione statica. Ora, a partire dagli stessi dati, viene presentata anche la gestione dinamica ed ottimizzata dello stesso flusso di auto in una tabella di confronto riassuntiva. Si può da subito notare come in tutte le corsie vi sia un miglioramento, in alcuni casi molto marcato, soprattutto per quelle collegate ai Generator, in altri lieve ma comunque presente. Dai grafici successivi è ancora più facile comprendere quanto si sta affermando, e come non vi sia alcun motivo per non implementare un meccanismo di gestione come quello presentato in questo lavoro di tesi per gestire al meglio gli incroci cittadini.
\begin{table}[H]
\centering
\begin{tabular}{|C{2.2cm}|C{2.2cm}|C{2.9cm}|C{2.9cm}|C{2.9cm}|}
\hline
\textbf{Corsia} &
\textbf{Tipologia} &
\textbf{Max numero auto in coda} &
\textbf{Tempo di attesa medio} &
\textbf{Numero di auto totale} \\\hline
\multirow{3}{2.2cm}{\centering \footnotesize{[1 - 1] Ovest - Dritto-Destra}} &
\footnotesize{Generator (Statico)} &
12 &
42.81 &
2550 \\\cline{2-5}
&
\footnotesize{Generator (Dinamico)} &
8 &
16.42 &
2551 \\\hline
\multirow{3}{2.2cm}{\centering \footnotesize{[1 - 1] Ovest - Sinistra}} &
\footnotesize{Generator (Statico)} &
15 &
49.27 &
2647 \\\cline{2-5}
&
\footnotesize{Generator (Dinamico)} &
10 &
18.94 &
2650 \\\hline
\multirow{3}{2.2cm}{\centering \footnotesize{[1 - 1] Nord - Dritto-Destra}} &
\footnotesize{Generator (Statico)} &
11 &
38.01 &
2301 \\\cline{2-5}
&
\footnotesize{Generator (Dinamico)} &
9 &
20.77 &
2301 \\\hline
\multirow{3}{2.2cm}{\centering \footnotesize{[1 - 1] Nord - Sinistra}} &
\footnotesize{Generator (Statico)} &
12 &
37.63 &
2402 \\\cline{2-5}
&
\footnotesize{Generator (Dinamico)} &
7 &
18.96 &
2402 \\\hline
\multirow{3}{2.2cm}{\centering \footnotesize{[1 - 1] Est - Dritto-Destra}} &
\footnotesize{Terminator (Statico)} &
9 &
21.96 &
2346 \\\cline{2-5}
&
\footnotesize{Terminator (Dinamico)} &
8 &
19.76 &
2346 \\\hline
\multirow{3}{2.2cm}{\centering \footnotesize{[1 - 1] Est - Sinistra}} &
\footnotesize{Intermedio (Statico)} &
10 &
21.95 &
2402 \\\cline{2-5}
&
\footnotesize{Intermedio (Dinamico)} &
10 &
21.39 &
2401 \\\hline
\multirow{3}{2.2cm}{\centering \footnotesize{[1 - 1] Sud - Dritto-Destra}} &
\footnotesize{Terminator (Statico)} &
9 &
23.79 &
2427 \\\cline{2-5}
&
\footnotesize{Terminator (Dinamico)} &
6 &
26.42 &
2430 \\\hline
\multirow{3}{2.2cm}{\centering \footnotesize{[1 - 1] Sud - Sinistra}} &
\footnotesize{Terminator (Statico)} &
10 &
27.48 &
2416 \\\cline{2-5}
&
\footnotesize{Terminator (Dinamico)} &
10 &
25.44 &
2417 \\\hline
\multirow{3}{2.2cm}{\centering \footnotesize{[1 - 2] Ovest - Dritto-Destra}} &
\footnotesize{Intermedio (Statico)} &
10 &
52.48 &
2510 \\\cline{2-5}
&
\footnotesize{Intermedio (Dinamico)} &
9 &
16.77 &
2513 \\\hline
\end{tabular}
\caption{Confronto fra gestione statica e dinamica di un cluster di nove incroci - pt.1}
\label{table:keytable}
\end{table}
\newpage
\begin{table}[H]
\centering
\begin{tabular}{|C{2.2cm}|C{2.2cm}|C{2.9cm}|C{2.9cm}|C{2.9cm}|}
\hline
\textbf{Corsia} &
\textbf{Tipologia} &
\textbf{Max numero auto in coda} &
\textbf{Tempo di attesa medio} &
\textbf{Numero di auto totale} \\\hline
\multirow{3}{2.2cm}{\centering \footnotesize{[1 - 2] Ovest - Sinistra}} &
\footnotesize{Terminator (Statico)} &
9 &
45.73 &
2264 \\\cline{2-5}
&
\footnotesize{Terminator (Dinamico)} &
7 &
17.87 &
2265 \\\hline
\multirow{3}{2.2cm}{\centering \footnotesize{[1 - 2] Nord - Dritto-Destra}} &
\footnotesize{Generator (Statico)} &
11 &
38.84 &
2342 \\\cline{2-5}
&
\footnotesize{Generator (Dinamico)} &
7 &
20.26 &
2345 \\\hline
\multirow{3}{2.2cm}{\centering \footnotesize{[1 - 2] Nord - Sinistra}} &
\footnotesize{Generator (Statico)} &
9 &
35.20 &
2219 \\\cline{2-5}
&
\footnotesize{Generator (Dinamico)} &
9 &
19.06 &
2223 \\\hline
\multirow{3}{2.2cm}{\centering \footnotesize{[1 - 2] Est - Dritto-Destra}} &
\footnotesize{Intermedio (Statico)} &
10 &
23.95 &
2306 \\\cline{2-5}
&
\footnotesize{Intermedio (Dinamico)} &
9 &
19.13 &
2306 \\\hline
\multirow{3}{2.2cm}{\centering \footnotesize{[1 - 2] Est - Sinistra}} &
\footnotesize{Intermedio (Statico)} &
7 &
22.83 &
2329 \\\cline{2-5}
&
\footnotesize{Intermedio (Dinamico)} &
8 &
21.67 &
2329 \\\hline
\multirow{3}{2.2cm}{\centering \footnotesize{[1 - 2] Sud - Dritto-Destra}} &
\footnotesize{Terminator (Statico)} &
10 &
52.39 &
2488 \\\cline{2-5}
&
\footnotesize{Terminator (Dinamico)} &
7 &
25.08 &
2488 \\\hline
\multirow{3}{2.2cm}{\centering \footnotesize{[1 - 2] Sud - Sinistra}} &
\footnotesize{Intermedio (Statico)} &
10 &
46.14 &
2407 \\\cline{2-5}
&
\footnotesize{Intermedio (Dinamico)} &
7 &
23.67 &
2407 \\\hline
\multirow{3}{2.2cm}{\centering \footnotesize{[1 - 3] Ovest - Dritto-Destra}} &
\footnotesize{Terminator (Statico)} &
10 &
46.32 &
2330 \\\cline{2-5}
&
\footnotesize{Terminator (Dinamico)} &
6 &
15.55 &
2332 \\\hline
\multirow{3}{2.2cm}{\centering \footnotesize{[1 - 3] Ovest - Sinistra}} &
\footnotesize{Terminator (Statico)} &
10 &
47.46 &
2311 \\\cline{2-5}
&
\footnotesize{Terminator (Dinamico)} &
7 &
17.63 &
2312 \\\hline
\end{tabular}
\caption{Confronto fra gestione statica e dinamica di un cluster di nove incroci - pt.2}
\label{table:keytable}
\end{table}
\newpage
\begin{table}[H]
\centering
\begin{tabular}{|C{2.2cm}|C{2.2cm}|C{2.9cm}|C{2.9cm}|C{2.9cm}|}
\hline
\textbf{Corsia} &
\textbf{Tipologia} &
\textbf{Max numero auto in coda} &
\textbf{Tempo di attesa medio} &
\textbf{Numero di auto totale} \\\hline
\multirow{3}{2.2cm}{\centering \footnotesize{[1 - 3] Nord - Dritto-Destra}} &
\footnotesize{Generator (Statico)} &
8 &
36.24 &
2143 \\\cline{2-5}
&
\footnotesize{Generator (Dinamico)} &
8 &
18.12 &
2144 \\\hline
\multirow{3}{2.2cm}{\centering \footnotesize{[1 - 3] Nord - Sinistra}} &
\footnotesize{Generator (Statico)} &
11 &
41.70 &
2333 \\\cline{2-5}
&
\footnotesize{Generator (Dinamico)} &
7 &
19.27 &
2337 \\\hline
\multirow{3}{2.2cm}{\centering \footnotesize{[1 - 3] Est - Dritto-Destra}} &
\footnotesize{Generator (Statico)} &
21 &
44.85 &
2660 \\\cline{2-5}
&
\footnotesize{Generator (Dinamico)} &
9 &
18.57 &
2660 \\\hline
\multirow{3}{2.2cm}{\centering \footnotesize{[1 - 3] Est - Sinistra}} &
\footnotesize{Generator (Statico)} &
9 &
38.46 &
2123 \\\cline{2-5}
&
\footnotesize{Generator (Dinamico)} &
6 &
20.55 &
2124 \\\hline
\multirow{3}{2.2cm}{\centering \footnotesize{[1 - 3] Sud - Dritto-Destra}} &
\footnotesize{Terminator (Statico)} &
10 &
25.90 &
2416 \\\cline{2-5}
&
\footnotesize{Terminator (Dinamico)} &
7 &
25.26 &
2416 \\\hline
\multirow{3}{2.2cm}{\centering \footnotesize{[1 - 3] Sud - Sinistra}} &
\footnotesize{Intermedio (Statico)} &
10 &
30.20 &
2450 \\\cline{2-5}
&
\footnotesize{Intermedio (Dinamico)} &
8 &
22.30 &
2451 \\\hline
\multirow{3}{2.2cm}{\centering \footnotesize{[2 - 1] Ovest - Dritto-Destra}} &
\footnotesize{Generator (Statico)} &
11 &
40.34 &
2358 \\\cline{2-5}
&
\footnotesize{Generator (Dinamico)} &
10 &
17.64 &
2359 \\\hline
\multirow{3}{2.2cm}{\centering \footnotesize{[2 - 1] Ovest - Sinistra}} &
\footnotesize{Generator (Statico)} &
7 &
36.91 &
2052 \\\cline{2-5}
&
\footnotesize{Generator (Dinamico)} &
6 &
17.56 &
2055 \\\hline
\multirow{3}{2.2cm}{\centering \footnotesize{[2 - 1] Nord - Dritto-Destra}} &
\footnotesize{Intermedio (Statico)} &
10 &
48.55 &
2410 \\\cline{2-5}
&
\footnotesize{Intermedio (Dinamico)} &
8 &
19.43 &
2415 \\\hline
\end{tabular}
\caption{Confronto fra gestione statica e dinamica di un cluster di nove incroci - pt.3}
\label{table:keytable}
\end{table}
\newpage
\begin{table}[H]
\centering
\begin{tabular}{|C{2.2cm}|C{2.2cm}|C{2.9cm}|C{2.9cm}|C{2.9cm}|}
\hline
\textbf{Corsia} &
\textbf{Tipologia} &
\textbf{Max numero auto in coda} &
\textbf{Tempo di attesa medio} &
\textbf{Numero di auto totale} \\\hline
\multirow{3}{2.2cm}{\centering \footnotesize{[2 - 1] Nord - Sinistra}} &
\footnotesize{Intermedio (Statico)} &
9 &
45.89 &
2470 \\\cline{2-5}
&
\footnotesize{Intermedio (Dinamico)} &
9 &
19.01 &
2471 \\\hline
\multirow{3}{2.2cm}{\centering \footnotesize{[2 - 1] Est - Dritto-Destra}} &
\footnotesize{Terminator (Statico)} &
9 &
48.96 &
2401 \\\cline{2-5}
&
\footnotesize{Terminator (Dinamico)} &
7 &
19.08 &
2402 \\\hline
\multirow{3}{2.2cm}{\centering \footnotesize{[2 - 1] Est - Sinistra}} &
\footnotesize{Intermedio (Statico)} &
10 &
48.44 &
2401 \\\cline{2-5}
&
\footnotesize{Intermedio (Dinamico)} &
8 &
21.41 &
2403 \\\hline
\multirow{3}{2.2cm}{\centering \footnotesize{[2 - 1] Sud - Dritto-Destra}} &
\footnotesize{Intermedio (Statico)} &
10 &
49.39 &
2640 \\\cline{2-5}
&
\footnotesize{Intermedio (Dinamico)} &
10 &
23.81 &
2640 \\\hline
\multirow{3}{2.2cm}{\centering \footnotesize{[2 - 1] Sud - Sinistra}} &
\footnotesize{Terminator (Statico)} &
10 &
48.50 &
2485 \\\cline{2-5}
&
\footnotesize{Terminator (Dinamico)} &
10 &
23.44 &
2484 \\\hline
\multirow{3}{2.2cm}{\centering \footnotesize{[2 - 2] Ovest - Dritto-Destra}} &
\footnotesize{Intermedio (Statico)} &
10 &
24.46 &
2347 \\\cline{2-5}
&
\footnotesize{Intermedio (Dinamico)} &
7 &
17.86 &
2348 \\\hline
\multirow{3}{2.2cm}{\centering \footnotesize{[2 - 2] Ovest - Sinistra}} &
\footnotesize{Intermedio (Statico)} &
8 &
23.63 &
2288 \\\cline{2-5}
&
\footnotesize{Intermedio (Dinamico)} &
6 &
19.14 &
2294 \\\hline
\multirow{3}{2.2cm}{\centering \footnotesize{[2 - 2] Nord - Dritto-Destra}} &
\footnotesize{Intermedio (Statico)} &
10 &
22.66 &
2222 \\\cline{2-5}
&
\footnotesize{Intermedio (Dinamico)} &
7 &
21.42 &
2223 \\\hline
\multirow{3}{2.2cm}{\centering \footnotesize{[2 - 2] Nord - Sinistra}} &
\footnotesize{Intermedio (Statico)} &
9 &
24.08 &
2329 \\\cline{2-5}
&
\footnotesize{Intermedio (Dinamico)} &
6 &
19.26 &
2329 \\\hline
\end{tabular}
\caption{Confronto fra gestione statica e dinamica di un cluster di nove incroci - pt.4}
\label{table:keytable}
\end{table}
\newpage
\begin{table}[H]
\centering
\begin{tabular}{|C{2.2cm}|C{2.2cm}|C{2.9cm}|C{2.9cm}|C{2.9cm}|}
\hline
\textbf{Corsia} &
\textbf{Tipologia} &
\textbf{Max numero auto in coda} &
\textbf{Tempo di attesa medio} &
\textbf{Numero di auto totale} \\\hline
\multirow{3}{2.2cm}{\centering \footnotesize{[2 - 2] Est - Dritto-Destra}} &
\footnotesize{Intermedio (Statico)} &
10 &
50.58 &
2510 \\\cline{2-5}
&
\footnotesize{Intermedio (Dinamico)} &
10 &
19.24 &
2510 \\\hline
\multirow{3}{2.2cm}{\centering \footnotesize{[2 - 2] Est - Sinistra}} &
\footnotesize{Intermedio (Statico)} &
10 &
54.43 &
2550 \\\cline{2-5}
&
\footnotesize{Intermedio (Dinamico)} &
10 &
21.53 &
2551 \\\hline
\multirow{3}{2.2cm}{\centering \footnotesize{[2 - 2] Sud - Dritto-Destra}} &
\footnotesize{Intermedio (Statico)} &
9 &
43.98 &
2364 \\\cline{2-5}
&
\footnotesize{Intermedio (Dinamico)} &
8 &
23.88 &
2364 \\\hline
\multirow{3}{2.2cm}{\centering \footnotesize{[2 - 2] Sud - Sinistra}} &
\footnotesize{Intermedio (Statico)} &
10 &
45.38 &
2477 \\\cline{2-5}
&
\footnotesize{Intermedio (Dinamico)} &
7 &
24.10 &
2478 \\\hline
\multirow{3}{2.2cm}{\centering \footnotesize{[2 - 3] Ovest - Dritto-Destra}} &
\footnotesize{Terminator (Statico)} &
8 &
23.55 &
2273 \\\cline{2-5}
&
\footnotesize{Terminator (Dinamico)} &
7 &
20.56 &
2273 \\\hline
\multirow{3}{2.2cm}{\centering \footnotesize{[2 - 3] Ovest - Sinistra}} &
\footnotesize{Intermedio (Statico)} &
9 &
23.86 &
2295 \\\cline{2-5}
&
\footnotesize{Intermedio (Dinamico)} &
7 &
18.32 &
2295 \\\hline
\multirow{3}{2.2cm}{\centering \footnotesize{[2 - 3] Nord - Dritto-Destra}} &
\footnotesize{Intermedio (Statico)} &
10 &
49.60 &
2510 \\\cline{2-5}
&
\footnotesize{Intermedio (Dinamico)} &
8 &
21.42 &
2512 \\\hline
\multirow{3}{2.2cm}{\centering \footnotesize{[2 - 3] Nord - Sinistra}} &
\footnotesize{Terminator (Statico)} &
10 &
52.28 &
2473 \\\cline{2-5}
&
\footnotesize{Terminator (Dinamico)} &
8 &
19.20 &
2476 \\\hline
\multirow{3}{2.2cm}{\centering \footnotesize{[2 - 3] Est - Dritto-Destra}} &
\footnotesize{Generator (Statico)} &
12 &
34.79 &
2076 \\\cline{2-5}
&
\footnotesize{Generator (Dinamico)} &
7 &
19.64 &
2079 \\\hline
\end{tabular}
\caption{Confronto fra gestione statica e dinamica di un cluster di nove incroci - pt.5}
\label{table:keytable}
\end{table}
\newpage
\begin{table}[H]
\centering
\begin{tabular}{|C{2.2cm}|C{2.2cm}|C{2.9cm}|C{2.9cm}|C{2.9cm}|}
\hline
\textbf{Corsia} &
\textbf{Tipologia} &
\textbf{Max numero auto in coda} &
\textbf{Tempo di attesa medio} &
\textbf{Numero di auto totale} \\\hline
\multirow{3}{2.2cm}{\centering \footnotesize{[2 - 3] Est - Sinistra}} &
\footnotesize{Generator (Statico)} &
16 &
48.05 &
2444 \\\cline{2-5}
&
\footnotesize{Generator (Dinamico)} &
9 &
22.67 &
2446 \\\hline
\multirow{3}{2.2cm}{\centering \footnotesize{[2 - 3] Sud - Dritto-Destra}} &
\footnotesize{Intermedio (Statico)} &
10 &
60.28 &
2594 \\\cline{2-5}
&
\footnotesize{Intermedio (Dinamico)} &
9 &
24.95 &
2594 \\\hline
\multirow{3}{2.2cm}{\centering \footnotesize{[2 - 3] Sud - Sinistra}} &
\footnotesize{Intermedio (Statico)} &
10 &
56.55 &
2659 \\\cline{2-5}
&
\footnotesize{Intermedio (Dinamico)} &
8 &
25.20 &
2659 \\\hline
\multirow{3}{2.2cm}{\centering \footnotesize{[3 - 1] Ovest - Dritto-Destra}} &
\footnotesize{Generator (Statico)} &
14 &
43.94 &
2403 \\\cline{2-5}
&
\footnotesize{Generator (Dinamico)} &
10 &
23.32 &
2404 \\\hline
\multirow{3}{2.2cm}{\centering \footnotesize{[3 - 1] Ovest - Sinistra}} &
\footnotesize{Generator (Statico)} &
39 &
76.84 &
2719 \\\cline{2-5}
&
\footnotesize{Generator (Dinamico)} &
11 &
26.29 &
2720 \\\hline
\multirow{3}{2.2cm}{\centering \footnotesize{[3 - 1] Nord - Dritto-Destra}} &
\footnotesize{Terminator (Statico)} &
10 &
27.69 &
2353 \\\cline{2-5}
&
\footnotesize{Terminator (Dinamico)} &
8 &
26.70 &
2354 \\\hline
\multirow{3}{2.2cm}{\centering \footnotesize{[3 - 1] Nord - Sinistra}} &
\footnotesize{Intermedio (Statico)} &
10 &
22.04 &
2305 \\\cline{2-5}
&
\footnotesize{Intermedio (Dinamico)} &
10 &
27.68 &
2305 \\\hline
\multirow{3}{2.2cm}{\centering \footnotesize{[3 - 1] Est - Dritto-Destra}} &
\footnotesize{Terminator (Statico)} &
10 &
56.07 &
2728 \\\cline{2-5}
&
\footnotesize{Terminator (Dinamico)} &
10 &
28.95 &
2728 \\\hline
\multirow{3}{2.2cm}{\centering \footnotesize{[3 - 1] Est - Sinistra}} &
\footnotesize{Terminator (Statico)} &
10 &
52.92 &
2678 \\\cline{2-5}
&
\footnotesize{Terminator (Dinamico)} &
10 &
25.59 &
2679 \\\hline
\end{tabular}
\caption{Confronto fra gestione statica e dinamica di un cluster di nove incroci - pt.6}
\label{table:keytable}
\end{table}
\newpage
\begin{table}[H]
\centering
\begin{tabular}{|C{2.2cm}|C{2.2cm}|C{2.9cm}|C{2.9cm}|C{2.9cm}|}
\hline
\textbf{Corsia} &
\textbf{Tipologia} &
\textbf{Max numero auto in coda} &
\textbf{Tempo di attesa medio} &
\textbf{Numero di auto totale} \\\hline
\multirow{3}{2.2cm}{\centering \footnotesize{[3 - 1] Sud - Dritto-Destra}} &
\footnotesize{Generator (Statico)} &
26 &
85.20 &
2791 \\\cline{2-5}
&
\footnotesize{Generator (Dinamico)} &
12 &
30.69 &
2792 \\\hline
\multirow{3}{2.2cm}{\centering \footnotesize{[3 - 1] Sud - Sinistra}} &
\footnotesize{Generator (Statico)} &
25 &
57.81 &
2683 \\\cline{2-5}
&
\footnotesize{Generator (Dinamico)} &
8 &
27.07 &
2685 \\\hline
\multirow{3}{2.2cm}{\centering \footnotesize{[3 - 2] Ovest - Dritto-Destra}} &
\footnotesize{Intermedio (Statico)} &
10 &
31.00 &
2500 \\\cline{2-5}
&
\footnotesize{Intermedio (Dinamico)} &
10 &
22.11 &
2502 \\\hline
\multirow{3}{2.2cm}{\centering \footnotesize{[3 - 2] Ovest - Sinistra}} &
\footnotesize{Intermedio (Statico)} &
10 &
40.48 &
2498 \\\cline{2-5}
&
\footnotesize{Intermedio (Dinamico)} &
9 &
21.93 &
2499 \\\hline
\multirow{3}{2.2cm}{\centering \footnotesize{[3 - 2] Nord - Dritto-Destra}} &
\footnotesize{Terminator (Statico)} &
10 &
26.28 &
2437 \\\cline{2-5}
&
\footnotesize{Terminator (Dinamico)} &
9 &
22.54 &
2441 \\\hline
\multirow{3}{2.2cm}{\centering \footnotesize{[3 - 2] Nord - Sinistra}} &
\footnotesize{Intermedio (Statico)} &
10 &
24.65 &
2380 \\\cline{2-5}
&
\footnotesize{Intermedio (Dinamico)} &
10 &
27.66 &
2381 \\\hline
\multirow{3}{2.2cm}{\centering \footnotesize{[3 - 2] Est - Dritto-Destra}} &
\footnotesize{Intermedio (Statico)} &
10 &
51.64 &
2554 \\\cline{2-5}
&
\footnotesize{Intermedio (Dinamico)} &
8 &
23.70 &
2556 \\\hline
\multirow{3}{2.2cm}{\centering \footnotesize{[3 - 2] Est - Sinistra}} &
\footnotesize{Terminator (Statico)} &
10 &
53.59 &
2548 \\\cline{2-5}
&
\footnotesize{Terminator (Dinamico)} &
9 &
25.26 &
2548 \\\hline
\multirow{3}{2.2cm}{\centering \footnotesize{[3 - 2] Sud - Dritto-Destra}} &
\footnotesize{Generator (Statico)} &
31 &
80.12 &
2943 \\\cline{2-5}
&
\footnotesize{Generator (Dinamico)} &
10 &
29.43 &
2944 \\\hline
\end{tabular}
\caption{Confronto fra gestione statica e dinamica di un cluster di nove incroci - pt.7}
\label{table:keytable}
\end{table}
\newpage
\begin{table}[H]
\centering
\begin{tabular}{|C{2.2cm}|C{2.2cm}|C{2.9cm}|C{2.9cm}|C{2.9cm}|}
\hline
\textbf{Corsia} &
\textbf{Tipologia} &
\textbf{Max numero auto in coda} &
\textbf{Tempo di attesa medio} &
\textbf{Numero di auto totale} \\\hline
\multirow{3}{2.2cm}{\centering \footnotesize{[3 - 2] Sud - Sinistra}} &
\footnotesize{Generator (Statico)} &
16 &
41.90 &
2094 \\\cline{2-5}
&
\footnotesize{Generator (Dinamico)} &
9 &
26.82 &
2094 \\\hline
\multirow{3}{2.2cm}{\centering \footnotesize{[3 - 3] Ovest - Dritto-Destra}} &
\footnotesize{Terminator (Statico)} &
10 &
40.42 &
2383 \\\cline{2-5}
&
\footnotesize{Terminator (Dinamico)} &
10 &
24.54 &
2383 \\\hline
\multirow{3}{2.2cm}{\centering \footnotesize{[3 - 3] Ovest - Sinistra}} &
\footnotesize{Intermedio (Statico)} &
8 &
21.69 &
2318 \\\cline{2-5}
&
\footnotesize{Intermedio (Dinamico)} &
10 &
23.95 &
2318 \\\hline
\multirow{3}{2.2cm}{\centering \footnotesize{[3 - 3] Nord - Dritto-Destra}} &
\footnotesize{Terminator (Statico)} &
10 &
25.92 &
2216 \\\cline{2-5}
&
\footnotesize{Terminator (Dinamico)} &
10 &
29.21 &
2220 \\\hline
\multirow{3}{2.2cm}{\centering \footnotesize{[3 - 3] Nord - Sinistra}} &
\footnotesize{Terminator (Statico)} &
8 &
23.73 &
2232 \\\cline{2-5}
&
\footnotesize{Terminator (Dinamico)} &
10 &
24.43 &
2236 \\\hline
\multirow{3}{2.2cm}{\centering \footnotesize{[3 - 3] Est - Dritto-Destra}} &
\footnotesize{Generator (Statico)} &
47 &
104.77 &
2936 \\\cline{2-5}
&
\footnotesize{Generator (Dinamico)} &
13 &
26.29 &
2936 \\\hline
\multirow{3}{2.2cm}{\centering \footnotesize{[3 - 3] Est - Sinistra}} &
\footnotesize{Generator (Statico)} &
17 &
64.44 &
2847 \\\cline{2-5}
&
\footnotesize{Generator (Dinamico)} &
12 &
26.46 &
2846 \\\hline
\multirow{3}{2.2cm}{\centering \footnotesize{[3 - 3] Sud - Dritto-Destra}} &
\footnotesize{Generator (Statico)} &
17 &
57.00 &
2559 \\\cline{2-5}
&
\footnotesize{Generator (Dinamico)} &
12 &
28.56 &
2560 \\\hline
\multirow{3}{2.2cm}{\centering \footnotesize{[3 - 3] Sud - Sinistra}} &
\footnotesize{Generator (Statico)} &
20 &
66.16 &
2837 \\\cline{2-5}
&
\footnotesize{Generator (Dinamico)} &
12 &
30.64 &
2837 \\\hline
\end{tabular}
\caption{Confronto fra gestione statica e dinamica di un cluster di nove incroci - pt.8}
\label{table:keytable}
\end{table}
\newpage
I seguenti grafici si riferiscono alla stessa corsia analizzata nel \textit{Capitolo 3}, ovvero quella più congestionata se gestita staticamente. Questa volta, però, viene riportato come variano nel tempo il numero di auto in coda, i tempi di attesa ed il numero di automobili che la attraversano, se a gestire il cluster è l'algoritmo dinamico presentato in questa tesi. Facendo un confronto con le figure del capitolo precedente, si nota che il paragone è impietoso per il cluster gestito staticamente, le cui performance non sono paragonabili a quelle raggiunte con l'applicazione dell'algoritmo.

\begin{figure}[H]
\centering
  \includegraphics[width=0.9\textwidth]{cap4congestionemax.pdf}
  \caption{Variazione nel tempo del numero di auto in coda, corsia più congestionata della simulazione se gestita staticamente}
  \label{fig:}
\end{figure}
\begin{figure}[H]
\centering
  \includegraphics[width=0.9\textwidth]{cap4congestionetempi.pdf}
  \caption{Tempi di attesa in funzione della fascia oraria, corsia più congestionata della simulazione se gestita staticamente}
  \label{fig:}
\end{figure}
\begin{figure}[H]
\centering
  \includegraphics[width=0.9\textwidth]{cap4congestionenumauto.pdf}
  \caption{Numero complessivo di auto, corsia più congestionata della simulazione se gestita staticamente}
  \label{fig:}
\end{figure}
\newpage
Possiamo infatti notare che il massimo numero di auto in coda passa da 47 a 13, calando drasticamente a parità di tutte le altre condizioni. Anche il tempo medio di attesa si riduce notevolmente, da \textit{104.77s} a \textit{26.29s}. 

Un'obiezione di un lettore di questa tesi potrebbe riguardare il fatto che, per snellire questa corsia in particolare, altre subiscano un congestionamento. È bene dunque precisare che, complice l'introduzione del meccanismo di gestione della starvation presentato nei capitoli precedenti, nessuna strada viene trascurata. Dalla tabella, riassuntiva di quanto detto in questo capitolo si può notare, infatti, che ogni corsia, non importa di quale incrocio faccia parte o quale sia la sua tipoligia, viene amministrata in maniera ottimale, con risultati sempre migliori rispetto ad una gestione statica.
\newline

















\chapter*{Conclusione}

In conclusione, con questo lavoro di tesi si è voluta dimostrare l'efficacia dell'algoritmo proposto, che certamente potrà essere migliorato dai futuri lettori, ma che rappresenta un punto di partenza per città più efficienti, meno caotiche e soprattutto meno inquinate, a fronte di un riutilizzo delle tecnologie esistenti e senza ripensare gli incroci cittadini.

Visti i risultati incoraggianti, il passo successivo è quello di realizzare un prototipo funzionante, con delle telecamere da collegare ai semafori di un incrocio, che inviino le loro acquisizioni ad un calcolatore locale, il quale mediante algoritmi di Computer Vision riesca a contare le automobili per ogni corsia, implementando così l'algoritmo e schedulando l'impianto semaforico.

Fatto questo, l'idea è quella di includere tutti i tipi di giunzione, non solo quella a raso a quattro bracci, creando un algoritmo universale che, mediante AI e Machine Learning, riesca autonomamente a riconoscere il tipo di giunzione che sta gestendo e come schedulare al meglio i semafori che ne fanno parte.


Dunque la speranza è che questo progetto rappresenti un punto di partenza per futuri sviluppi, e che possa trovare sbocchi applicativi reali, compiendo, nel suo piccolo, un ulteriore passo verso la trasformazione digitale delle città in cui tutti viviamo.


% Inserimento bibliografia
\printbibliography

\end{document}